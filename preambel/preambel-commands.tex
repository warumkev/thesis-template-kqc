%%% Packages for LaTeX - programming
\usepackage{xspace}
\usepackage{ifthen}
\usepackage{ifpdf}

%%% Internal Commands
\makeatletter

% Hilfsbefehle für bedingte Ausführung
\providecommand{\IfDefined}[2]{%
\ifcsname #1\endcsname #2 \fi
}

\providecommand{\IfElseDefined}[3]{%
\ifcsname #1\endcsname #2 \else #3 \fi
}

% Frontmatter, Mainmatter, Backmatter definieren
\@ifundefined{frontmatter}{%
   \newcommand{\frontmatter}{\pagenumbering{roman}}
}{}
\@ifundefined{mainmatter}{%
   \newif\if@mainmatter\@mainmattertrue
   \newcommand{\mainmatter}{\pagenumbering{arabic}\setcounter{page}{1}}
}{}
\@ifundefined{backmatter}{%
   \newcommand{\backmatter}{\pagenumbering{roman}}
}{}

% KI-Deklaration
\newcommand*{\printGenerativeAIDeclaration}{%
\ifcsdef{declareUseOfGenerativeAITool}{%
\ifthenelse{\equal{\lang}{ngerman}}%
{%
\section*{Erklärung zur Nutzung von KI-basierten Werkzeugen}
Bei der Erstellung der vorliegenden Arbeit habe ich KI-basierte Anwendungen bzw.~Werkzeuge als Hilfsmittel verwendet.
Die verwendeten Tools (\declareUseOfGenerativeAITool) sowie Art und Umfang der Nutzung sind im Anhang A
„Verzeichnis der KI-Nutzung" (siehe \autoref{ch:ki-nutzung}) tabellarisch dokumentiert.

Ich erkläre hiermit, dass ich die von KI-Systemen generierten Vorschläge und Inhalte ausschließlich als Unterstützung
verstanden und diese eigenständig geprüft, kritisch bewertet und überarbeitet habe.
Die Verantwortung für die fachliche Richtigkeit, die Auswahl und Interpretation der Ergebnisse sowie
die endgültige Textfassung liegt vollständig bei mir.

\vspace{1em}
{\footnotesize Diese Erklärung folgt den Empfehlungen der Deutschen Forschungsgemeinschaft (DFG).}
}
{%
\section*{Declaration of the use of Generative AI}
During the preparation of this work I used \declareUseOfGenerativeAITool\ to improve readability and language.
I reviewed and edited the content as needed and take full responsibility for the content of this thesis.
}
}{}
}

% Geschlechter-Disclaimer
\newcommand*\printGenderDisclaimer{%
\ifthenelse{\equal{\lang}{ngerman}}%
{%
\section*{Gender-Disclaimer}
\thispagestyle{empty}
In dieser Arbeit wird aus Gründen der besseren Lesbarkeit das generische Maskulinum verwendet.
}
{%
\section*{Gender Disclaimer}
For reasons of better readability, the generic masculine is used in this work.
}
}

% Unabhängigkeitserklärung
\newcommand*\printDeclarationOfIndependence{%
\clearpage
\ifthenelse{\equal{\lang}{ngerman}}%
{%
\section*{Erklärung der Selbstständigkeit}
\thispagestyle{empty}
Hiermit versichere ich, die vorliegende Arbeit selbstständig verfasst und keine anderen als die angegebenen Quellen und Hilfsmittel benutzt zu haben.

\vspace{4\baselineskip}
\noindent Frankfurt am Main, den \today \hfill \theAuthor
\vspace{4\baselineskip}
}
{%
\section*{Declaration of Independence}
\thispagestyle{empty}
I hereby declare that I have composed the present work independently and have not used any sources or aids other than those cited.

\vspace{4\baselineskip}
\noindent Frankfurt am Main, on \today \hfill \theAuthor
\vspace{4\baselineskip}
}
}

% Metadaten-Befehle
\let\oldauthor\author
\renewcommand{\author}[1]{\oldauthor{#1}\newcommand{\theAuthor}{#1}}

\let\oldtitle\title
\renewcommand{\title}[1]{\oldtitle{#1}\newcommand{\theTitle}{#1}}

\newcommand*{\academicTitle}[1]{\newcommand{\theAcedemicTitle}{#1}}
\newcommand*{\thesis}[1]{\newcommand{\theThesis}{#1}}
\newcommand*{\firstReferee}[1]{\newcommand{\theFirstReferee}{#1}}
\newcommand*{\secondReferee}[1]{\newcommand{\theSecondReferee}{#1}}
\newcommand*{\studentID}[1]{\newcommand{\theStudentID}{#1}}
\newcommand*{\studentAddress}[1]{\newcommand{\theStudentAddress}{#1}}

\makeatother