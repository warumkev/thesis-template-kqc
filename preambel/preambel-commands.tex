%%% Packages for LaTeX - programming
\usepackage{xspace}
\usepackage{ifthen}
\usepackage{ifpdf}

%%% Internal Commands
\makeatletter

% Hilfsbefehle für bedingte Ausführung
\providecommand{\IfDefined}[2]{%
\ifcsname #1\endcsname #2 \fi%
}

\providecommand{\IfElseDefined}[3]{%
\ifcsname #1\endcsname #2 \else #3 \fi%
}

% Frontmatter, Mainmatter, Backmatter definieren
\@ifundefined{frontmatter}{%
   \newcommand{\frontmatter}{\pagenumbering{roman}}%
}{}
\@ifundefined{mainmatter}{%
   \newif\if@mainmatter\@mainmattertrue
   \newcommand{\mainmatter}{\pagenumbering{arabic}\setcounter{page}{1}}
}{}
\@ifundefined{backmatter}{%
   \newcommand{\backmatter}{\pagenumbering{roman}}
}{}

% KI-Deklaration (Text in Anhang: content/Z-Anhang-declaration.tex)
% Das eigentliche Deklarationstext wurde in den Anhang verschoben, damit er nur
% einmal an der gewünschten Stelle erscheint. Dieses Makro bleibt bestehen,
% damit bestehende Aufrufe im Dokument weiterhin funktionieren, gibt jedoch
% keinen Text mehr aus.
\newcommand*{\printGenerativeAIDeclaration}{}%

% Geschlechter-Disclaimer
\newcommand*\printGenderDisclaimer{%
\ifthenelse{\equal{\lang}{ngerman}}%
{%
\section*{Gender-Disclaimer}%
\thispagestyle{empty}
In dieser Arbeit wird aus Gründen der besseren Lesbarkeit das generische Maskulinum verwendet.
}
{%
\section*{Gender Disclaimer}%
For reasons of better readability, the generic masculine is used in this work.
}
}

% Unabhängigkeitserklärung
\newcommand*\printDeclarationOfIndependence{%
\clearpage
\ifthenelse{\equal{\lang}{ngerman}}%
{%
\section*{Erklärung der Selbstständigkeit}%
\thispagestyle{empty}
Hiermit versichere ich, die vorliegende Arbeit selbstständig verfasst und keine anderen als die angegebenen Quellen und Hilfsmittel benutzt zu haben.

\vspace{4\baselineskip}
\noindent Frankfurt am Main, den \today \hfill \theAuthor
\vspace{4\baselineskip}
}
{%
\section*{Declaration of Independence}%
\thispagestyle{empty}
I hereby declare that I have composed the present work independently and have not used any sources or aids other than those cited.

\vspace{4\baselineskip}
\noindent Frankfurt am Main, on \today \hfill \theAuthor
\vspace{4\baselineskip}
}
}

% Metadaten-Befehle
\let\oldauthor\author
\renewcommand{\author}[1]{\oldauthor{#1}\newcommand{\theAuthor}{#1}}

\let\oldtitle\title
\renewcommand{\title}[1]{\oldtitle{#1}\newcommand{\theTitle}{#1}}

\newcommand*{\academicTitle}[1]{\newcommand{\theAcedemicTitle}{#1}}
\newcommand*{\thesis}[1]{\newcommand{\theThesis}{#1}}
\newcommand*{\firstReferee}[1]{\newcommand{\theFirstReferee}{#1}}
\newcommand*{\secondReferee}[1]{\newcommand{\theSecondReferee}{#1}}
\newcommand*{\studentID}[1]{\newcommand{\theStudentID}{#1}}
\newcommand*{\studentAddress}[1]{\newcommand{\theStudentAddress}{#1}}

\makeatother