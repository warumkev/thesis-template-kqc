\chapter{Einführung}
\index{Agentic AI}
\index{Software Engineering}

\section{Motivation und Relevanz}

Software Engineering erlebt derzeit einen Paradigmenwechsel: \emph{agentic AI} \textendash{} also KI-Systeme, die Ziele verstehen, Pläne erstellen, Werkzeuge verwenden und Ergebnisse eigenständig verifizieren \textendash{} ergänzt klassische Automatisierung um adaptive, mehrschrittige Problemlösung.\footnote{Der Begriff \emph{Agentic AI} wird derzeit von führenden Forschungsteams geprägt und bezieht sich auf Systeme, die über mehrere Schritte selbstständig komplexe Aufgaben bewältigen können.} Für Informatikstudierende der Fachrichtung \enquote{Software Engineering mit agentic AI} eröffnet dies neue Architektur- und Methodikfragen: Wie entwirft man robuste Agenten-Workflows? Wie orchestriert man Tool-Nutzung, Gedächtnis und Langkontext? Und wie integriert man Sicherheit, Nachvollziehbarkeit und Tests in agentische Systeme?\footnote{Praktische Orchestrierung bedeutet hier die Koordination von Planungsschritten, Werkzeugaufrufen und Gedächtniszugriffen in einer strukturierten Abfolge.}

\section{Problemstellung}

Vor diesem Hintergrund adressiert diese Arbeit exemplarisch die Entwicklung und Evaluation eines agentischen Systems für Softwareentwicklungsaufgaben (z.\,B. Refactoring, Code-Review, Generierung von Tests). Zentrale Fragen sind:

\begin{itemize}
  \item Wie lassen sich Agenten-Policies (Planen, Tool-Aufrufe, Selbstkritik) systematisch modellieren?
  \item Wie werden externe Werkzeuge (VCS, CI, linters, Issue-Tracker) sicher und nachvollziehbar eingebunden?
  \item Welche Metriken messen Fortschritt, Qualität und Sicherheit realistisch?
\end{itemize}

\section{Zielsetzung}

Die Ziele dieser Arbeit sind auf die Spezialisierung \enquote{Software Engineering mit agentic AI} zugeschnitten:

\begin{enumerate}
  \item Analyse des Forschungsstands zu agentischen Architekturen und Orchestrierungsframeworks
  \item Entwurf einer referenzierbaren Agentenarchitektur für Software-Engineering-Aufgaben
  \item Implementierung eines prototypischen Agenten mit Werkzeuganbindung und Gedächtnis
  \item Evaluation anhand reproduzierbarer Benchmarks (Qualität, Kosten, Laufzeit, Sicherheit)
\end{enumerate}

\section{Abgrenzung des Themas}

Die Arbeit fokussiert Agenten für Softwareentwicklungsaufgaben. Nicht im Fokus sind z.\,B. Reinforcement Learning from Human Feedback (RLHF) im Detail, Trainingsmethoden auf Rohdaten oder proprietäre Interna von Foundation Models. Ebenso werden Domänen außerhalb der Softwareentwicklung (z.\,B. Robotik) nicht betrachtet.

\begin{itemize}
  \item Zu komplexe Spezialfälle, die für diese Arbeit nicht relevant sind
  \item Historische Entwicklungen vor einem bestimmten Zeitpunkt
  \item Randbereiche, die außerhalb des Fokus liegen
\end{itemize}

\section{Aufbau der Arbeit}

Die restliche Arbeit gliedert sich wie folgt:

\begin{itemize}
  \item \textbf{Kapitel~\ref{ch:hintergrund}}: Grundlagen zu agentischen Systemen (Tool-Nutzung, Planung, Gedächtnis) und relevante Arbeiten.
  \item \textbf{Kapitel~\ref{ch:konzept}}: Referenzierbare Agentenarchitektur (Zustandsmodell, Policies, Schnittstellen).
  \item \textbf{Kapitel~\ref{ch:realisierung}}: Implementierung mit Beispiel-Listings und Ergebnisse.
  \item \textbf{Kapitel~\ref{ch:abschluss}}: Zusammenfassung und Ausblick auf zukünftige Arbeiten.
\end{itemize}

