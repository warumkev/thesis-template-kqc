% !TEX root = ../Thesis.tex

\chapter{Verzeichnis der KI-Nutzung}
\label{ch:ki-nutzung}

Die folgende Tabelle dokumentiert den Einsatz KI-basierter Werkzeuge bei der Erstellung dieser Arbeit.
Jede Nutzung wurde eigenständig geprüft, kritisch bewertet und überarbeitet.

\vspace{2em}
\noindent\textbf{Hinweis:} Alle generierten Vorschläge wurden eigenständig geprüft, fachlich bewertet und gegebenenfalls angepasst oder verworfen. Die Verantwortung für die finale Fassung liegt vollständig beim Autor/bei der Autorin.

\begin{landscape}
\begin{longtable}{|p{3cm}|p{3.5cm}|p{6cm}|p{5cm}|p{2.5cm}|}
\hline
\textbf{KI-Tool} & \textbf{Zweck} & \textbf{Kontext/Eingangstext} & \textbf{Generierte Ausgabe} & \textbf{Kapitel} \\
\hline
\endfirsthead

\multicolumn{5}{c}%
{{\tablename\ \thetable{} --- Fortsetzung}} \\
\hline
\textbf{KI-Tool} & \textbf{Zweck} & \textbf{Kontext/Eingangstext} & \textbf{Generierte Ausgabe} & \textbf{Kapitel} \\
\hline
\endhead

\hline
\endfoot

\hline
\endlastfoot

% Beispieleinträge (bitte anpassen oder löschen)
Anthropic Claude Opus 4.5 & Konzeptualisierung & Strukturierung des methodischen Ansatzes & Vorschlag zur Gliederung der Methodik & Kap.~3.1 \\
\hline

Anthropic Claude Opus 4.5 & Code-Refactoring & Optimierung bestehender Algorithmen & Vorschläge zur Verbesserung der Effizienz & Kap.~4.3 \\
\hline

Cohere Command A & RAG/Agentik & Entwurf eines Retrieval-Flows & Vorschlag für Tool-Aufrufe und Prompt-Struktur & Kap.~3.2 \\
\hline

DeepL Agent & Übersetzung & Deutsche Formulierung für Abstract & Professionelle englische Übersetzung & Abstract \\
\hline

GitHub Copilot & Code-Vervollständigung & Python-Funktion für Datenverarbeitung & Vorschläge für Funktionsimplementierung & Kap.~4.2 \\
\hline

Google Gemini 3 Pro & Literaturrecherche & Zusammenfassung aktueller Forschungsarbeiten & Überblick über Stand der Technik & Kap.~2.1 \\
\hline

Mistral Large 3 & Technische Zusammenfassung & Auswertung von API-/RFC-Dokumenten & Kompakte Zusammenfassung der Kernaussagen & Kap.~2.4 \\
\hline

OpenAI GPT-5.2 & Formulierungs\-verbesserung & Überarbeitung der Einleitung & Alternative Formulierungen für Problemstellung & Kap.~1.2, S.~5 \\
\hline

xAI Grok 4.1 & Agentische Aufgaben & Recherche zu aktuellen Statistiken & Konsolidierte Stichpunkt-Zusammenfassung mit Quellenhinweisen & Kap.~2.3 \\
\hline

% Weitere Einträge hier einfügen...
% Tool & Zweck & Kontext & Ausgabe & Kapitel \\
% \hline

\end{longtable}
\end{landscape}

\chapter{Zusätzliche Materialien}

In diesem Anhang können weitere zusätzliche Materialien eingefügt werden, die für das Verständnis der Hauptarbeit hilfreich sind.

\section{Weitere Tabellen und Daten}

Tabellarische Daten, die zu umfangreich für die Hauptkapitel sind.

\section{Quellcode und Implementierungsdetails}

Längere Quellcode-Listings oder detaillierte Implementierungen.

\section{Ergänzende Berechnungen}

Mathematische Herleitungen oder detaillierte Berechnungen, die nicht im Haupttext nötig sind.
