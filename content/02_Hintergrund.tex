% !TEX root = ../Bachelor-Thesis.tex

\chapter{Theoretischer Hintergrund}
\label{ch:hintergrund}

\section{Grundkonzepte}

Dieses Kapitel führt in Grundbegriffe agentischer Systeme ein und bildet die theoretische Basis für Konzept und Implementierung \cite{smith2020example}.

\subsection{Agentic AI: Begriffe und Bausteine}

Kernbausteine agentischer Systeme sind (i) \emph{Zustand} (Kontext, Ziele, Erinnerungen), (ii) \emph{Policy} (Planung, Aktion, Selbstkritik), (iii) \emph{Werkzeuge} (Funktionen/\enquote{Tools} wie Code-Ausführung, Websuche, VCS) und (iv) \emph{Gedächtnis} (episodisch/semantisch). Orchestrierung umfasst Planung (z.\,B. ReAct), Tool-Auswahl, Fehlerbehandlung und Reflexion.

\subsection{Etablierte Methoden und Frameworks}

Relevante Muster sind \emph{Chain-of-Thought}, \emph{ReAct}, \emph{Tree-of-Thought}, \emph{Plan-and-Execute} sowie Graph-basierte Orchestrierung. Praxisnahe Frameworks bieten Funktionen für Tool-Anbindung, Speicherschichten und Kontrollfluss \cite{doe2019research}.

\section{Verwandte Arbeiten}

Es existiert umfangreiche Literatur zu Tool-Nutzung, Langkontext, Agentenplanung und Auswertung. Benchmarks adressieren Code-Qualität, Erfolgsrate (\enquote{task success}), Kosten und Sicherheit.

\subsection{Ansatz A: ReAct-ähnliche Planung}

ReAct verbindet reasoning und acting: Das Modell plant Teilschritte, ruft Tools auf und reflektiert. Vorteile: gute Transparenz, einfache Implementierung. Grenzen: längere Latenzen, potentielle Halluzinationen.

\subsection{Ansatz B: Graph-/Workflow-basierte Orchestrierung}

Graphen erlauben robuste Kontrollflüsse (Retry, Branching, Parallelisierung), klare Zustandsübergänge und bessere Testbarkeit. Grenzen: initialer Modellierungsaufwand, Overhead \cite{lee2018conference}.

\section{Vergleich und Bewertung}

Tabelle \ref{tab:agententypen} vergleicht typische Agententypen. Für Software-Engineering-Aufgaben erweisen sich werkzeugnutzende Agenten mit Reflexion als besonders geeignet. Daraus leitet sich die in Kapitel \ref{ch:konzept} entwickelte Architektur ab.

\begin{table}[ht]
\centering
\caption{Agententypen und Fähigkeiten (Beispieltabelle)}
\label{tab:agententypen}
\begin{tabular}{L{3.5cm}L{9.5cm}}
\toprule
\cellcolor{headercolor}\textcolor{white}{\textbf{Agententyp}} & \cellcolor{headercolor}\textcolor{white}{\textbf{Kurzbeschreibung}} \\
\midrule
Reaktiver Agent & Einzelne Schritte ohne längerfristige Planung; geeignet für klar definierte Aufgaben mit geringer Komplexität. \\
\rowcolor{rowcolorgray}
Planender Agent & Erstellt und aktualisiert Pläne; gut für mehrschrittige Aufgaben mit Abhängigkeiten. \\
Werkzeugnutzender Agent & Ruft externe Tools (z.\,B. Linter, Tests, VCS) auf; hohe praktische Nützlichkeit im Software Engineering. \\
\rowcolor{rowcolorgray}
Mehragentensystem & Rollenbasierte Zusammenarbeit (z.\,B. Reviewer, Coder, Tester); skaliert bei größeren Projekten. \\
\bottomrule
\end{tabular}
\end{table}

\section{Forschungslücke}

Basierend auf der Analyse ergeben sich u.\,a. folgende Lücken:

\begin{itemize}
  \item Fehlende Referenzarchitekturen für agentische SE-Workflows mit robustem Tooling
  \item Skalierungs- und Kostenfragen bei langem Kontext und vielen Tool-Aufrufen
  \item Realistische Metriken und Evals für Qualität, Sicherheit und Nachvollziehbarkeit
\end{itemize}

Diese Arbeit trägt dazu bei, diese Lücken zu schließen.

