% !TEX root = ../Thesis.tex

\chapter*{Abkürzungsverzeichnis}
\label{ch:abkuerzungen}%
\addcontentsline{toc}{chapter}{Abkürzungsverzeichnis}

% Abkürzungstabelle
\rowcolors{2}{rowcolorgray}{}
\begin{longtable}{@{} l p{11cm} @{}}
\toprule
\cellcolor{headercolor}\textcolor{white}{\textbf{Abkürzung}} & \cellcolor{headercolor}\textcolor{white}{\textbf{Bedeutung}} \\
\midrule
\endhead%
\midrule
\multicolumn{2}{r@{}}{\textit{Fortsetzung auf nächster Seite}} \\
\endfoot%
\bottomrule
\endlastfoot%
%
ACI & Agent-Computer-Interface \\
AI & Artificial Intelligence (Künstliche Intelligenz) \\
API & Application Programming Interface (Anwendungsprogrammierschnittstelle) \\
APPS & Automated Programming Progress Standard \\
AST & Abstract Syntax Tree (Abstrakter Syntaxbaum) \\
CD & Continuous Delivery (Kontinuierliche Auslieferung) \\
CI & Continuous Integration \\
CLI & Command Line Interface (Befehlszeilenschnittstelle) \\
CoT & Chain-of-Thought (Gedankenkette) \\
DB & Datenbank (Database) \\
DFG & Deutsche Forschungsgemeinschaft \\
E/A & Eingabe/Ausgabe (Input/Output) \\
FAQ & Frequently Asked Questions (Häufig gestellte Fragen) \\
GPT & Generative Pre-trained Transformer \\
GPU & Graphics Processing Unit (Grafik-Verarbeitungseinheit) \\
HTTP & HyperText Transfer Protocol \\
I/O & Input/Output (Eingabe/Ausgabe) \\
IDE & Integrated Development Environment (Integrierte Entwicklungsumgebung) \\
JSON & JavaScript Object Notation \\
KB & Knowledge Base (Wissensdatenbank) \\
KG & Knowledge Graph (Wissensgraph) \\
KI & Künstliche Intelligenz \\
LLM & Large Language Model (Großes Sprachmodell) \\
LOC & Lines of Code (Codezeilen) \\
MBPP & Mostly Basic Python Problems \\
ML & Machine Learning (Maschinelles Lernen) \\
NLP & Natural Language Processing (Natürlichsprachverarbeitung) \\
PII & Personally Identifiable Information (Personenbezogene Daten) \\
PR & Pull Request \\
PRD & Product Requirement Document (Produktanforderungsdokument) \\
QA & Quality Assurance (Qualitätssicherung) \\
QS & Qualitätssicherung \\
RAG & Retrieval-Augmented Generation (Abruf-gestützte Generierung) \\
RAM & Random Access Memory (Arbeitsspeicher) \\
ReAct & Reasoning and Acting (Denken und Handeln) \\
REST & Representational State Transfer \\
RFC & Request for Comments (IETF-Standarddokumente) \\
RLHF & Reinforcement Learning from Human Feedback \\
SE & Software Engineering (Software-Entwicklung) \\
SMT & Satisfiability Modulo Theories \\
SQL & Structured Query Language (Strukturierte Abfragesprache) \\
SSD & Solid State Drive (Festkörperspeicher) \\
SSH & Secure Shell \\
VCS & Version Control System (Versionskontrollsystem) \\
XML & Extensible Markup Language \\
YAML & YAML Ain't Markup Language \\
\end{longtable}
