% !TEX root = ../Thesis.tex

\chapter*{Abkürzungsverzeichnis}
\label{ch:abkuerzungen}

% Apply alternating colors to body rows starting with the first data row
\rowcolors{1}{rowcolorgray}{}

% Reduce tabcolsep and font size locally to avoid overfull boxes in longtable
\begingroup
		\setlength{\tabcolsep}{4pt}
		small
		% Narrow second column slightly to avoid minor overfull boxes
		\begin{longtable}{L{2cm}L{11.5cm}}
\toprule
\cellcolor{headercolor}\textcolor{white}{\textbf{Abkürzung}} & \cellcolor{headercolor}\textcolor{white}{\textbf{Bedeutung}} \\
\midrule
\endhead
\midrule
\multicolumn{2}{r}{\cellcolor{white}\textit{Fortsetzung auf nächster Seite}} \\
\endfoot
\bottomrule
\endlastfoot
%
AI & Artificial Intelligence (Künstliche Intelligenz) \\
API & Application Programming Interface (Anwendungsprogrammierschnittstelle) \\
CI & Continuous Integration \\
LLM & Large Language Model (Großes Sprachmodell) \\
ReAct & Reasoning and Acting (Denken und Handeln) \\
RAG & Retrieval-Augmented Generation (Abruf-gestützte Generierung) \\
VCS & Version Control System (Versionskontrollsystem) \\
CLI & Command Line Interface (Befehlszeilenschnittstelle) \\
IDE & Integrated Development Environment (Integrierte Entwicklungsumgebung) \\
SE & Software Engineering (Software-Entwicklung) \\
JSON & JavaScript Object Notation \\
RFC & Request for Comments (IETF-Standarddokumente) \\
YAML & YAML Ain't Markup Language \\
SQL & Structured Query Language (Strukturierte Abfragesprache) \\
SSH & Secure Shell \\
GPU & Graphics Processing Unit (Grafik-Verarbeitungseinheit) \\
HTTP & HyperText Transfer Protocol \\
REST & Representational State Transfer \\
XML & Extensible Markup Language \\
KB & Knowledge Base (Wissensdatenbank) \\
KG & Knowledge Graph (Wissensgraph) \\
ML & Machine Learning (Maschinelles Lernen) \\
NLP & Natural Language Processing (Natürlichsprachverarbeitung) \\
	\end{longtable}
\endgroup
