%!TEX root = ../Thesis.tex
% Erklärung zur geschlechtergerechten Sprache
% Basierend auf den Empfehlungen der Thüringer Hochschulen

\chapter*{Hinweis zur geschlechtergerechten Sprache}
\addcontentsline{toc}{chapter}{Hinweis zur geschlechtergerechten Sprache}

\noindent
In dieser Arbeit wird auf eine geschlechtergerechte Sprache geachtet. Aus Gründen der besseren Lesbarkeit wird in dieser Arbeit eine Mischung verschiedener Formen geschlechtergerechter Sprache verwendet:

\begin{itemize}
    \item \textbf{Geschlechtsneutrale Formulierungen} (z.\,B. Studierende, Mitarbeitende, Person)
    \item \textbf{Paarformen} (z.\,B. Entwicklerinnen und Entwickler, Nutzerinnen und Nutzer)
    \item \textbf{Genderstern} (z.\,B. Benutzer*innen) zur Sichtbarmachung aller Geschlechter
\end{itemize}

\noindent
Alle verwendeten Personenbezeichnungen gelten gleichermaßen für alle Geschlechter. Diese Arbeit orientiert sich an den Empfehlungen des Leitfadens \enquote{Sag's doch gleich! Geschlechtergerechte Sprache an Thüringer Hochschulen} des Thüringer Kompetenznetzwerks Gleichstellung (TKG).
