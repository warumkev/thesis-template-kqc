% !TEX root = ../Thesis.tex

%% Leitfragen für das Abstract:
%% - Motivation: Warum ist das Thema relevant?
%% - Fragestellung: Was untersucht die Arbeit?
%% - Methode: Wie wurde untersucht (Ansatz, Techniken)?
%% - Ergebnisse: Zu welchen Schlussfolgerungen kam die Forschung?
%% - Relevanz: Was trägt die Arbeit zum bestehenden Wissen bei?
%%
%% Format: 150-250 Wörter, prägnant und eigenständig verständlich

\chapter*{Kurzfassung}
\addcontentsline{toc}{chapter}{Kurzfassung}

% Reduziere den Absatzabstand lokal, damit alles auf eine Seite passt
\begingroup
\setlength{\parskip}{0.5em}
\textbf{Kontext und Motivation:} Large Language Models (LLMs) haben sich von reinen Textgeneratoren zu agentischen Systemen entwickelt, die selbstständig planen, Werkzeuge nutzen und mehrschrittige Aufgaben bewältigen können. Im Software Engineering eröffnet das neue Möglichkeiten für die Automatisierung komplexer Workflows wie Code-Review, Testgenerierung und Refactoring.

\textbf{Zielsetzung:} Die Arbeit untersucht die Entwicklung und Evaluation agentischer Architekturen für Software-Engineering-Aufgaben. Zentrale Forschungsfragen sind: (1) Wie lassen sich robuste Agentenstrategien für SE-Workflows systematisch modellieren? (2) Welche Architekturprinzipien ermöglichen sichere und nachvollziehbare Werkzeugintegration? (3) Mit welchen Metriken kann die Leistungsfähigkeit agentischer Systeme realistisch bewertet werden?

\textbf{Methodik:} Basierend auf einer systematischen Literaturanalyse wird eine Referenzarchitektur entwickelt, die Planung (ReAct-Pattern), Werkzeugnutzung (Linter, Tests, VCS), episodisches Gedächtnis und Sicherheitsmechanismen kombiniert. Ein prototypisches System wurde in Python implementiert und anhand reproduzierbarer Benchmarks evaluiert.

\textbf{Ergebnisse:} Die Evaluation zeigt, dass die entwickelte Architektur in kontrollierten Szenarien eine Erfolgsrate von \pct{73} bei automatisiertem Refactoring erreicht, während die durchschnittliche Bearbeitungszeit um \pct{45} reduziert wird. Reflexionsmechanismen verbessern die Robustheit bei fehlerhaften Werkzeugausgaben um \pct{28}. Kostenanalysen belegen, dass optimierte Kontextverwaltung die Token-Nutzung um bis zu \pct{40} senken kann.

\textbf{Beitrag:} Die Arbeit liefert eine praxisnahe Referenzarchitektur, konkrete Implementierungsrichtlinien und evaluierte Metriken für agentische SE-Systeme. Sie zeigt sowohl technische Potenziale als auch kritische Limitierungen auf und diskutiert gesellschaftliche Implikationen der Automatisierung.%
\par
\endgroup

% Englisches Abstract (Standard in der Informatik)
\chapter*{Abstract}
\addcontentsline{toc}{chapter}{Abstract}

\begingroup
\setlength{\parskip}{0.5em}
\textbf{Context and Motivation:} Large Language Models (LLMs) have evolved from pure text generators to agentic systems capable of planning, using tools, and handling multi-step tasks autonomously. In software engineering, this opens up new possibilities for automating complex workflows such as code review, test generation, and refactoring.

\textbf{Objective:} This thesis investigates the development and evaluation of agentic architectures for software engineering tasks. Central research questions are: (1)~How can robust agent policies for SE workflows be systematically modeled? (2)~Which architectural principles enable secure and traceable tool integration? (3)~Which metrics can realistically assess the performance of agentic systems?

\textbf{Methodology:} Based on a systematic literature review, a reference architecture is developed that combines planning (ReAct pattern), tool usage (linters, tests, VCS), episodic memory, and security mechanisms. A prototypical system was implemented in Python and evaluated using reproducible benchmarks.

\textbf{Results:} The evaluation shows that the developed architecture achieves a success rate of \pct{73} in automated refactoring in controlled scenarios, while reducing average processing time by \pct{45}. Reflection mechanisms improve robustness against erroneous tool outputs by \pct{28}. Cost analyses demonstrate that optimized context management can reduce token usage by up to \pct{40}.

\textbf{Contribution:} The work provides a practical reference architecture, concrete implementation guidelines, and evaluated metrics for agentic SE systems. It highlights both technical potentials and critical limitations and discusses societal implications of automation.%
\par
\endgroup

\cleardoublepage
