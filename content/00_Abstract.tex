% !TEX root = ../Thesis.tex

%% Leitfragen für das Abstract:
%% - Motivation: Warum ist das Thema relevant?
%% - Fragestellung: Was untersucht die Arbeit?
%% - Methode: Wie wurde untersucht (Ansatz, Techniken)?
%% - Ergebnisse: Zu welchen Schlussfolgerungen kam die Forschung?
%% - Relevanz: Was trägt die Arbeit zum bestehenden Wissen bei?
%%
%% Format: 150-250 Wörter, prägnant und eigenständig verständlich

\chapter*{Kurzfassung}
\addcontentsline{toc}{chapter}{Kurzfassung}

\textbf{Kontext und Motivation:} Large Language Models (LLMs) haben sich von reinen Textgeneratoren zu agentischen Systemen entwickelt, die selbstständig planen, Werkzeuge nutzen und mehrschrittige Aufgaben bewältigen können. Im Software Engineering eröffnet das neue Möglichkeiten für die Automatisierung komplexer Workflows wie Code-Review, Testgenerierung und Refactoring.

\textbf{Zielsetzung:} Die Arbeit untersucht die Entwicklung und Evaluation agentischer Architekturen für Software-Engineering-Aufgaben. Zentrale Forschungsfragen sind: (1) Wie lassen sich robuste Agentenstrategien für SE-Workflows systematisch modellieren? (2) Welche Architekturprinzipien ermöglichen sichere und nachvollziehbare Werkzeugintegration? (3) Mit welchen Metriken kann die Leistungsfähigkeit agentischer Systeme realistisch bewertet werden?

\textbf{Methodik:} Basierend auf einer systematischen Literaturanalyse wird eine Referenzarchitektur entwickelt, die Planung (ReAct-Pattern), Werkzeugnutzung (Linter, Tests, VCS), episodisches Gedächtnis und Sicherheitsmechanismen kombiniert. Ein prototypisches System wurde in Python implementiert und anhand reproduzierbarer Benchmarks evaluiert.

\textbf{Ergebnisse:} Die Evaluation zeigt, dass die entwickelte Architektur in kontrollierten Szenarien eine Erfolgsrate von \pct{73} bei automatisiertem Refactoring erreicht, während die durchschnittliche Bearbeitungszeit um \pct{45} reduziert wird. Reflexionsmechanismen verbessern die Robustheit bei fehlerhaften Werkzeugausgaben um \pct{28}. Kostenanalysen belegen, dass optimierte Kontextverwaltung die Token-Nutzung um bis zu \pct{40} senken kann.

\textbf{Beitrag:} Die Arbeit liefert eine praxisnahe Referenzarchitektur, konkrete Implementierungsrichtlinien und evaluierte Metriken für agentische SE-Systeme. Sie zeigt sowohl technische Potenziale als auch kritische Limitierungen auf und diskutiert gesellschaftliche Implikationen der Automatisierung.

\cleardoublepage
