% !TEX root = ../Thesis.tex

%% Leitfragen für das Abstract:
%% - Motivation: Warum ist dieses Thema relevant?
%% - Fragestellung: Was untersucht die Arbeit?
%% - Methode: Wie wurde untersucht (Ansatz, Techniken)?
%% - Ergebnisse: Zu welchen Schlussfolgerungen kam die Forschung?
%% - Relevanz: Was trägt die Arbeit zum bestehenden Wissen bei?
%%
%% Format: 150-250 Wörter, prägnant und eigenständig verständlich

\chapter*{Kurzfassung}
\addcontentsline{toc}{chapter}{Kurzfassung}

Diese Abschlussarbeit beschäftigt sich mit der Optimierung von Prozessen im Bereich der Softwareentwicklung. Im Zentrum steht die Frage, wie etablierte Methoden durch moderne Technologien effizienter gestaltet werden können, ohne dabei bestehende Standards zu gefährden.

Die Motivation für diese Arbeit ergibt sich aus der wachsenden Komplexität von Softwareprojekten und dem gleichzeitigen Druck, Entwicklungszyklen zu verkürzen. Ein wichtiger Aspekt ist dabei die Sicherung der Code-Qualität trotz steigender Anforderungen.

Als Methode wurde eine Kombination aus Literaturrecherche, praktischer Implementierung eines Prototyps und evaluativen Studien durchgeführt. Die praktische Umsetzung basiert auf etablierten Frameworks und Best-Practices der Industrie.

Die wichtigsten Ergebnisse zeigen, dass durch gezielte Optimierungen eine Reduktion der Entwicklungszeit um durchschnittlich \pct{20} erreicht werden kann, während die Codequalität gleichbleibt oder sich sogar verbessert. Diese Resultate wurden durch mehrere Experimente validiert.

Insgesamt trägt diese Arbeit zu einem besseren Verständnis der Wechselwirkungen zwischen Effizienz und Qualität in der Softwareentwicklung bei und bietet praktische Ansätze für die Industrie.

\cleardoublepage
