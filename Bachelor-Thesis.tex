%% Dokumentenklasse (KOMA-Script) ==========================================
%% scrreprt: Report-Klasse mit Chaptern, ideal für Abschlussarbeiten
\documentclass[%
   final,%              % Finales Dokument (vs. draft)
   paper=a4,%           % DIN A4 Papierformat
   pagesize=auto,%      % PDF-Seitengröße automatisch einstellen
   fontsize=12pt,%      % Schriftgröße 12pt (Standard für Abschlussarbeiten)
   version=last%        % Neueste KOMA-Script Version
]{scrreprt}


% Text-Encoding für UTF-8 Umlaute und Sonderzeichen
\usepackage[utf8]{inputenc}

%% Spracheinstellung ========================================================
%% Options: ngerman (Deutsch, neue Rechtschreibung), english (Englisch)
\def\lang{ngerman}

%% KI-Tools Declaration (optional) ==========================================
%% Bitte ausfüllen, wenn KI-Tools bei der Erstellung verwendet wurden
%% \def\declareUseOfGenerativeAITool{ChatGPT~4}

%% Preambel laden ===========================================================
%%% === Textbody ==============================================================
\KOMAoptions{%
   DIV=11,%        % Größe des Satzspiegels (höher = größer)
   BCOR=5mm%       % Bindekorrektur für gebundene Arbeiten
}%

%%% === Überschriften ==========================================================
\KOMAoptions{%
   headings=big,%             % Große Überschriften mit viel Abstand
   headings=noappendixprefix,% Keine speziellen Anhang-Präfixe
   headings=nochapterprefix,%  % Keine Nummern-Präfixe bei Kapiteln
   headings=openright,%        % Kapitel starten auf rechter Seite
   numbers=noenddot%           % Kein Punkt nach Kapitelnummern
}%
\setcounter{secnumdepth}{3}

%%% === Seitenlayout ===========================================================
\KOMAoptions{%
   twoside=true,%      % Zweiseitige Darstellung (Buch-Standard)
   twocolumn=false,%   % Einspaltig
   headinclude=false,%
   footinclude=false,%
   headsepline=true,%  % Linie unter Kopfzeile
   cleardoublepage=empty%
}%

%%% === Absatzformatierung =====================================================
\KOMAoptions{%
   parskip=full%  % Deutsche Konvention: Keine Einrückung, Abstand zwischen Absätzen
}%

%%% === Table of Contents =====================================================
\setcounter{tocdepth}{3}
\KOMAoptions{%
   toc=indented,%
   toc=bib,%
   toc=noindex,%
   toc=nolistof%
}%

%%% === Lists of figures, tables etc. =========================================
\KOMAoptions{%
   listof=indented,%
   listof=chaptergapsmall,%
   listof=totoc%
}%

%%% === Bibliography ==========================================================
\KOMAoptions{%
   bibliography=openstyle,%
   bibliography=totoc%
}%

%%% === Index =================================================================
\KOMAoptions{%
   index=nottotoc%
}%

%%% === Titlepage =============================================================
\KOMAoptions{%
   titlepage=true%
}%

%%% === Miscellaneous =========================================================
\KOMAoptions{%
   footnotes=multiple%
}%
%%% ----------------------------------------------------------------------
%%% LaTeX-Präambel
%%% Hier werden alle Pakete geladen und Einstellungen vorgenommen.
%%% ----------------------------------------------------------------------

%%% 1. Basiskonfiguration & Hilfsbefehle
%%% ----------------------------------------------------------------------
%%% Packages for LaTeX - programming
\usepackage{xspace}
\usepackage{ifthen}
\usepackage{ifpdf}

%%% Internal Commands
\makeatletter

% Hilfsbefehle für bedingte Ausführung
\providecommand{\IfDefined}[2]{%
\ifcsname #1\endcsname #2 \fi%
}

\providecommand{\IfElseDefined}[3]{%
\ifcsname #1\endcsname #2 \else #3 \fi%
}

% Frontmatter, Mainmatter, Backmatter definieren
\@ifundefined{frontmatter}{%
   \newcommand{\frontmatter}{\pagenumbering{roman}}%
}{}
\@ifundefined{mainmatter}{%
   \newif\if@mainmatter\@mainmattertrue
   \newcommand{\mainmatter}{\pagenumbering{arabic}\setcounter{page}{1}}
}{}
\@ifundefined{backmatter}{%
   \newcommand{\backmatter}{\pagenumbering{roman}}
}{}

% KI-Deklaration (Text in Anhang: content/Z-Anhang-declaration.tex)
% Das eigentliche Deklarationstext wurde in den Anhang verschoben, damit er nur
% einmal an der gewünschten Stelle erscheint. Dieses Makro bleibt bestehen,
% damit bestehende Aufrufe im Dokument weiterhin funktionieren, gibt jedoch
% keinen Text mehr aus.
\newcommand*{\printGenerativeAIDeclaration}{}%

% Geschlechter-Disclaimer
\newcommand*\printGenderDisclaimer{%
\ifthenelse{\equal{\lang}{ngerman}}%
{%
\section*{Gender-Disclaimer}%
\thispagestyle{empty}
In dieser Arbeit wird aus Gründen der besseren Lesbarkeit das generische Maskulinum verwendet.
}
{%
\section*{Gender Disclaimer}%
For reasons of better readability, the generic masculine is used in this work.
}
}

% Unabhängigkeitserklärung
\newcommand*\printDeclarationOfIndependence{%
\clearpage
\ifthenelse{\equal{\lang}{ngerman}}%
{%
\section*{Erklärung der Selbstständigkeit}%
\thispagestyle{empty}
Hiermit versichere ich, die vorliegende Arbeit selbstständig verfasst und keine anderen als die angegebenen Quellen und Hilfsmittel benutzt zu haben.

\vspace{4\baselineskip}
\noindent Frankfurt am Main, den \today \hfill \theAuthor
\vspace{4\baselineskip}
}
{%
\section*{Declaration of Independence}%
\thispagestyle{empty}
I hereby declare that I have composed the present work independently and have not used any sources or aids other than those cited.

\vspace{4\baselineskip}
\noindent Frankfurt am Main, on \today \hfill \theAuthor
\vspace{4\baselineskip}
}
}

% Metadaten-Befehle
\let\oldauthor\author
\renewcommand{\author}[1]{\oldauthor{#1}\newcommand{\theAuthor}{#1}}

\let\oldtitle\title
\renewcommand{\title}[1]{\oldtitle{#1}\newcommand{\theTitle}{#1}}

\newcommand*{\academicTitle}[1]{\newcommand{\theAcedemicTitle}{#1}}
\newcommand*{\thesis}[1]{\newcommand{\theThesis}{#1}}
\newcommand*{\firstReferee}[1]{\newcommand{\theFirstReferee}{#1}}
\newcommand*{\secondReferee}[1]{\newcommand{\theSecondReferee}{#1}}
\newcommand*{\studentID}[1]{\newcommand{\theStudentID}{#1}}
\newcommand*{\studentAddress}[1]{\newcommand{\theStudentAddress}{#1}}

\makeatother

%%% Spracheinstellungen (Babel)
\ifthenelse{\equal{\lang}{ngerman}}
  {\usepackage[ngerman]{babel}}
  {\usepackage[english]{babel}}

%%% Farben
\usepackage[table]{xcolor}

%%% Bilder einbinden
\usepackage{graphicx}

%%% Querformat für einzelne Seiten
\usepackage{pdflscape}

%%% Rechnen mit LaTeX (wird oft intern benötigt)
\usepackage{calc}


%%% 2. Text & Typografie
%%% ----------------------------------------------------------------------

%%% Schrift-Encoding und Symbole
\usepackage[T1]{fontenc}
\usepackage{textcomp}

%%% Schriftarten laden (ausgelagert)
\input{preambel/Fonts}

%%% Zeilenabstand (1,5-fach)
\usepackage{setspace}
\onehalfspacing

%%% Mikrotypografie (Optischer Randausgleich etc.)
\usepackage[
   protrusion=true
]{microtype}

%%% Besserer Flattersatz (Linksbündig)
\usepackage{ragged2e}

%%% Intelligente "..." (\dots)
\usepackage{ellipsis}

%%% Textauszeichnungen (Unterstreichen, Durchstreichen, Sperren)
\usepackage[normalem]{ulem}
\usepackage{soul}

%%% URLs korrekt umbrechen
\usepackage{xurl}

%%% Anführungszeichen (passend zur Sprache)
\usepackage[babel, german=quotes, english=british, french=guillemets]{csquotes}
\SetBlockThreshold{2} % Ab 2 Zeilen wird ein Zitat als Block gesetzt


%%% 3. Layout & Seitenränder
%%% ----------------------------------------------------------------------

%%% Seitenränder einstellen (Geometry)
\usepackage{geometry}
\geometry{
   a4paper,
   portrait,
   ignoreall,      % Kopf-/Fußzeilen zählen nicht zum Rand
   heightrounded,  % Textblockhöhe an Zeilenraster anpassen
   hmargin={3.5cm, 2.5cm}, % Links / Rechts
   vmargin={3cm, 3cm},     % Oben / Unten
   bindingoffset=5mm,      % Bindekorrektur
   marginparwidth=0pt,
   twoside,
}

%%% Kopf- und Fußzeilen (KOMA-Script)
\usepackage[
   automark,
   pagestyleset=KOMA-Script,
   markcase=ignoreuppercase,
]{scrlayer-scrpage}

% Standard-Stile löschen
\clearmainofpairofpagestyles
\clearplainofpairofpagestyles

% Einstellung: Oben außen Kapitel/Sektion, Unten außen Seitenzahl
\ohead{\headmark}
\ofoot[\pagemark]{\pagemark}

% Linie unter der Kopfzeile
\KOMAoptions{headsepline=.4pt}
\addtokomafont{headsepline}{\color{black}}

% Breite von Kopf/Fuß an Text anpassen
\KOMAoptions{headwidth=text:0pt, footwidth=text:0pt}

% Schriftart für Kopf/Fuß
\setkomafont{pageheadfoot}{\normalfont\normalcolor\small\sffamily}
\setkomafont{pagenumber}{\bfseries\sffamily}

%%% Fußnoten
\usepackage[bottom, stable, multiple]{footmisc}
\deffootnote{1.5em}{1em}{\makebox[1.5em][l]{\thefootnotemark}} % Layout
\addtolength{\skip\footins}{\baselineskip} % Abstand Text <-> Fußnote

%%% Hurenkinder und Schusterjungen vermeiden (Einzelne Zeilen am Seitenanfang/-ende)
\clubpenalty = 10000
\widowpenalty = 10000
\displaywidowpenalty = 10000


%%% 4. Tabellen & Abbildungen
%%% ----------------------------------------------------------------------

%%% Tabellen-Optimierung
\usepackage{booktabs}  % Schöne Linien (\toprule, \midrule, \bottomrule)
\usepackage{multirow}  % Zellen über mehrere Zeilen
\usepackage{dcolumn}   % Ausrichtung an Komma/Punkt
\usepackage{ltxtable}  % Tabellen über mehrere Seiten (Longtable + Tabularx)
\usepackage{array}     % Erweiterte Spalten-Befehle
\usepackage{colortbl}  % Färbung von Zellen und Reihen

% Farben für alternierende Zeilen
\definecolor{rowcolorgray}{gray}{0.90}      % Hellgrau für jeden 2. Zeile
\definecolor{headercolor}{RGB}{120,150,180} % Matteres Blau für Header

% Erweiterte Spalten-Definition mit Padding
\newcolumntype{L}[1]{>{\raggedright\let\newline\\\arraybackslash\hspace{0pt}}m{#1}}
\newcolumntype{C}[1]{>{\centering\let\newline\\\arraybackslash\hspace{0pt}}m{#1}}
\newcolumntype{R}[1]{>{\raggedleft\let\newline\\\arraybackslash\hspace{0pt}}m{#1}}

% Tabellen-Einstellungen mit besserem Padding
\renewcommand{\arraystretch}{1.5}           % Zeilenhöhe erhöht
\setlength{\tabcolsep}{8pt}                 % Padding links/rechts von Zellen

%%% Abbildungen & Floats
\usepackage{float}     % Option [H] für feste Platzierung
\usepackage{flafter}   % Floats erst nach ihrer Referenz im Text
\usepackage[section]{placeins} % Floats bleiben in ihrer Section

%%% Unterabbildungen (Bild a, Bild b)
\usepackage{subcaption}

%%% TikZ für Diagramme
\usepackage{tikz}
\usetikzlibrary{arrows.meta,positioning,shapes,fit}

%%% Textumflossene Bilder
\usepackage{wrapfig}
\setlength{\intextsep}{0.5\baselineskip}

%%% Beschriftungen (Captions)
\usepackage{caption}
\captionsetup{
   margin = 10pt,
   font = {small,rm},
   labelfont = {small,bf},
   format = plain,
   indention = 0em,
   labelsep = colon,
   justification = RaggedRight,
   singlelinecheck = true,
   position = bottom
}
% Workaround für Warnungen
\DeclareCaptionOption{parskip}[]{}
\DeclareCaptionOption{parindent}[]{}

% Caption auch außerhalb von Floats erlauben
\usepackage{capt-of}

%%% Farbige Boxen (für Infoboxen etc.)
\usepackage[most]{tcolorbox}


%%% 5. Mathematik & Wissenschaft
%%% ----------------------------------------------------------------------

%%% Basispakete
\usepackage{amsmath}
\usepackage[fixamsmath,disallowspaces]{mathtools}

%%% ISO-Konforme Mathematik (Griechische Großbuchstaben kursiv etc.)
\usepackage{fixmath}

%%% Komma als Dezimaltrennzeichen im Mathe-Modus
\usepackage{icomma}

%%% Maßeinheiten & Zahlenformatierung (korrekter Abstand bei Prozentangaben)
\usepackage{siunitx}
\sisetup{
	detect-all = true,
	locale = DE,        % Deutsche Konventionen (Dezimal-Komma)
	per-mode = symbol   % Prozent als Symbol mit korrekt gesetztem Abstand
}
% Praktischer Prozent-Makro
\newcommand{\pct}[1]{\SI{#1}{\percent}}

%%% Zusätzliche Mathe-Symbole und Werkzeuge
\usepackage{braket}    % Bra-Ket Schreibweise
\usepackage{cancel}    % Durchstreichen
\usepackage{empheq}    % Hervorgehobene Gleichungen
\usepackage[Symbolsmallscale]{upgreek}
\usepackage[upmu]{gensymb}

%%% Epsilon-Symbol anpassen (gewohntes Aussehen)
\let\ORGvarepsilon=\varepsilon
\let\varepsilon=\epsilon
\let\epsilon=\ORGvarepsilon


%%% 6. Verzeichnisse & Referenzen
%%% ----------------------------------------------------------------------

%%% Literaturverzeichnis (BibLaTeX mit Biber)
\usepackage[style=alphabetic, backend=biber]{biblatex}
\addbibresource{bib/BibtexDatabase.bib}

%%% Index, Abkürzungen, Nomenklatur
% Use imakeidx for flexible index printing (supports columns)
\usepackage{imakeidx}
\usepackage[german]{nomencl}
\usepackage[printonlyused]{acronym}

%%% Intelligente Querverweise ("siehe Seite X")
\usepackage[ngerman]{varioref}

%%% PDF-Links und Bookmarks (Hyperref)
\usepackage[
   hidelinks,               % Keine bunten Rahmen
   linktocpage=true,        % Seitenzahlen im TOC verlinken
   bookmarks=true,          % Lesezeichen im PDF
   bookmarksopen=true,
   bookmarksnumbered=true,
   plainpages=false,        % Verhindert "duplicate page identifiers"
   pdfpagelabels=true,      % Korrekte Seitennummerierung (römisch/arabisch)
   pdftitle={},
   pdfauthor={},
]{hyperref}

%%% Links springen zum Bild, nicht zur Caption
\usepackage[figure,table]{hypcap}


%%% 7. Code & Listings
%%% ----------------------------------------------------------------------

%%% Quellcode-Darstellung
\usepackage{listings}
\usepackage{upquote} % Korrekte Anführungszeichen im Code

% Farben für Code-Highlighting
\definecolor{keywordblue}{RGB}{0,0,139}      % Dunkelblau
\definecolor{stringred}{RGB}{178,34,34}      % Firebrick (dunkelrot)
\definecolor{commentgreen}{RGB}{34,139,34}   % Forest Green
\definecolor{numbermauve}{RGB}{128,0,128}    % Dunkelviolett
\definecolor{bgcolor}{RGB}{245,245,245}      % Hellgrau

\lstset{
  basicstyle={\footnotesize\ttfamily},
	backgroundcolor=\color{bgcolor},
	breaklines=true,
	extendedchars=true,
	frame=tb,
	framexbottommargin=4pt,
	framexleftmargin=17pt,
	framexrightmargin=5pt,
	keywordstyle={\color{keywordblue}\bfseries},
	commentstyle={\color{commentgreen}\itshape},
	stringstyle={\color{stringred}},
	numbers=left,
	numbersep=5pt,
	numberstyle={\tiny\color{gray}},
	showspaces=false,
	showstringspaces=false,
	showtabs=false,
	tabsize=2,
	captionpos=b,
	abovecaptionskip=10pt,
	belowcaptionskip=10pt
}

% Lade vordefinierte Sprachen
\lstloadlanguages{Python,Java}

% TypeScript als Basis-Definition (nicht auf JavaScript basierend)
\lstdefinelanguage{TypeScript}{
	keywords={
		abstract,any,as,async,await,boolean,break,case,catch,class,const,constructor,continue,
		debugger,declare,default,delete,do,else,enum,export,extends,false,finally,for,from,
		function,get,global,if,import,in,instanceof,interface,is,keyof,let,module,namespace,
		never,new,null,of,package,private,protected,public,readonly,require,return,set,static,
		super,switch,symbol,this,throw,true,try,type,typeof,var,void,while,with,yield
	},
	sensitive=true,
	comment=[l]{//},
	morecomment=[s]{/*}{*/},
	string=[b]{"}, 
	string=[b]{'}, 
	string=[b]{`},
}

% Python-Highlighting
\lstdefinestyle{Python}{
	language=Python,
	keywordstyle={\color{keywordblue}\bfseries},
	commentstyle={\color{commentgreen}\itshape},
	stringstyle={\color{stringred}},
	breaklines=true
}

% TypeScript-Highlighting
\lstdefinestyle{TypeScript}{
	language=TypeScript,
	keywordstyle={\color{keywordblue}\bfseries},
	commentstyle={\color{commentgreen}\itshape},
	stringstyle={\color{stringred}},
	breaklines=true
}


%%% 8. Sonstiges
%%% ----------------------------------------------------------------------

%%% Listen (Aufzählungen) anpassen
\usepackage{enumitem}
\setlist{nosep} % Kompakte Listen: kein Abstand zwischen Items
% Alternative: \setlist{itemsep=0.5ex} für etwas Abstand

%%% Mehrspaltiger Text
\usepackage{multicol}

%%% Externe PDFs einbinden
\usepackage{pdfpages}

%%% Randnotizen
\usepackage{marginnote}

%%% TODO-Notizen
\setlength{\marginparwidth}{2cm}
\usepackage{todonotes}


%%% 9. Farben & Design
%%% ----------------------------------------------------------------------

% Farben definieren
\definecolor{sectioncolor}{RGB}{0, 0, 0}
\definecolor{textcolor}{RGB}{0, 0, 0}
\definecolor{shadecolor}{gray}{0.90}

% PDF-Link-Farben (werden bei hidelinks ignoriert, aber gut zu haben)
\definecolor{pdfurlcolor}{rgb}{0,0,0.6}
\definecolor{pdffilecolor}{rgb}{0.7,0,0}
\definecolor{pdflinkcolor}{rgb}{0,0,0.6}
\definecolor{pdfcitecolor}{rgb}{0,0,0.6}

% Schriftart für Überschriften (Sans-Serif)
\newcommand\SectionFontStyle{\sffamily}
\setkomafont{sectioning}{\SectionFontStyle}
\setkomafont{chapter}{\huge\SectionFontStyle}
\addtokomafont{sectioning}{\color{sectioncolor}}

%%% 10. Auszuführende Befehle
%%% ----------------------------------------------------------------------
% Enable index with two columns (print with \printindex in the document)
\makeindex[columns=2]
\IfDefined{makenomenclature}{\makenomenclature}
\renewcommand{\nomname}{Abkürzungsverzeichnis}


%% Metadaten der Arbeit ====================================================
%% Bitte alle Felder ausfüllen für korrekte Anzeige im PDF
\author{Maria Musterfrau}
\studentID{1234567}
\studentAddress{Musterstraße 1, 12345 Musterstadt}
\thesis{Bachelor-Thesis}
\title{Titel der Abschlussarbeit}
\academicTitle{Bachelor of Science}
\firstReferee{Prof. Dr. Klaus Quibeldey-Cirkel}
\secondReferee{Prof. Dr. Philipp Diebold}

%% Makros und benutzerdefinierte Befehle ====================================
% Placeholder macros/newcommands.tex
% Real macro definitions were not provided by the user.
% This file supplies safe, minimal no-op commands so the document can be built.

% Simple formatting helpers (no-ops or lightweight defaults)
\providecommand{\code}[1]{\texttt{#1}}
\providecommand{\TODO}[1]{\textbf{TODO: #1}}
\providecommand{\note}[1]{\textit{#1}}

% Ensure common metadata macros exist (if not defined elsewhere)
\providecommand{\theAuthor}{}
\providecommand{\theTitle}{}

% End of placeholder
%% ========================================================================
%% Benutzerdefinierte LaTeX-Befehle für Abschlussarbeiten
%% ========================================================================

%% Abbildungs- und Gleichungsverweise ====================================
\newcommand{\figureref}[1]{(Abbildung \ref{#1})}
\newcommand{\eqnref}[1]{(\ref{#1})}

%% Abbildungsüberschriften mit definierter Breite ========================
\newcommand{\wcaption}[2]{%
   \begin{minipage}{#1}%
   \caption{#2}%
   \end{minipage}%
}

%% Randnotizen mit optimierter Formatierung ===============================
\newcommand{\marginlabel}[1]{\marginnote{#1}}
\newcommand{\complex}{\mathbb{C}} % Complex
\newcommand{\real}{\mathbb{R}}    % Real
%\newcommand{\R}{\real}						% Real
%\newcommand{\N}{\mathbb{N}}
%\newcommand{\Z}{\mathbb{Z}}
\renewcommand{\L}{\mathcal{L}}
\newcommand{\N}{\mathcal{N}}
\newcommand{\R}{\mathcal{R}}
\newcommand{\D}{\mathcal{D}}
%
\newcommand{\Ham}{\mathcal{H}}    % Hamilton
\newcommand{\Prob}{\mathscr{P}}    % Hamilton
\newcommand{\unity}{\mathds{1}}   % Real
%

\newcommand\gammab{\gamma_\bot}
\newcommand\gammap{\gamma_\parallel}
\newcommand\gammai{\gamma_\text{int}}
\newcommand\gammae{\gamma_\text{ext}}

% -- New Operators --
\DeclareMathOperator{\rot}{rot}
\DeclareMathOperator{\grad}{grad}
%\DeclareMathOperator{\div}{div}
\renewcommand{\div}{\text{div}\,}
\DeclareMathOperator{\Tr}{Tr}
\DeclareMathOperator{\const}{const}
\DeclareMathOperator{\e}{e} 			% exponatial Function

% -- new symbols --
\newcommand{\laplace}{\Delta}
\newcommand{\dalembert}{\Box}

% -- new arrows --
\renewcommand{\leadsto}{\Longrightarrow}
\newcommand{\leftrightleadsto}{\Longleftrightarrow}


% -- Text subscripts--
\newcommand{\rel}{_\text{rel}}
%\newcommand{\st}{\text{st}}
%

% -- other --
\newcommand{\com}[2]{\underbrace{#1}_{\textrm{\scriptsize #2}}}
\newcommand{\with}[1]{\stackrel{\ref{#1}}{\Longrightarrow}}
%\newcommand{\unit}[1]{\,\textrm{#1}}

%\newcommand{\variance}[1]{\delta \mean{#1}^2}
\newcommand{\variance}[1]{(\Delta{#1})^2}
%\newcommand{\variance}[1]{\delta #1^2}

% -- Physics --------------------------------
\newcommand\op[1]{{\hat{\mathrm{#1}}}}  % Operator

\newcommand\expect[1]{\ensuremath{\left\langle{#1}\right\rangle}} %
%
\newcommand{\mean}[1]{\ensuremath{\overline{#1}}} % mean value
%
\newcommand{\state}[1]{\ensuremath{\ket{#1}}}
%
\newcommand\commutator[2]{\ensuremath{\mathinner{%
    \mathopen[\,#1,#2\,\mathclose]}}}
\newcommand{\Commutator}[2]{\ensuremath{\left[\,#1,#2\,\right]}}
\newcommand{\bigcommutator}[2]{\ensuremath{\bigl[\,#1,#2\,\bigr]}}
\newcommand{\Bigcommutator}[2]{\ensuremath{\Bigl[\,#1,#2\,\Bigr]}}
%
\newcommand\poisson[2]{\mathinner{%
    \mathopen\{#1,#2\mathclose\}}}
%

% -- Layout --------------------------------

\newcommand*{\dashfill}{\leavevmode\cleaders\hbox{-}\hfill\kern0pt}

\newcommand*{\midhrulefill}{
\leavevmode
\cleaders\hbox to 1ex{\raisebox{.5ex}{\rule{1ex}{.4pt}}}\hfill\kern0pt
}

%% Prozentangaben mit dünnem, geschütztem Abstand
%% Nutzung: \pct{20} ergibt "20 %" mit nicht trennbarem dünnem Leerzeichen
\newcommand{\pct}[1]{\SI{#1}{\percent}}


% Placeholder macros/TableCommands.tex
% Minimal table-related helpers so the document compiles even without the real file.

\providecommand{\TableHeader}[1]{\textbf{#1}}
\newenvironment{SimpleTable}[1][]{% begin: do nothing special
}{% end: nothing
}

% no-op column type if referenced
\providecolumntype{P}[1]{>{\raggedright\arraybackslash}p{#1}}

% End of placeholder
%% ========================================================================
%% Befehle für professionelle Tabellenformatierung
%% ========================================================================
%% Basierend auf: The LaTeX Companion, tabsatz.ps und Best Practices

%% Tabellenfarben ===========================================================
\IfPackageLoaded{xcolor}{
   \colorlet{tableheadcolor}{gray!25}
   \colorlet{tablerowcolor}{gray!10}
}

%% Spaltendefinitionen für Tabellen ========================================
%% Neue Spaltentypen mit verbesserter Textformatierung

% Linksgebündige Spalte mit definierter Breite und Flattersatz
\newcommand{\PreserveBackslash}[1]{\let\temp=\\#1\let\\=\temp}
\newcolumntype{v}[1]{>{\PreserveBackslash\RaggedRight\hspace{0pt}}p{#1}}

% Mittelaligned Spalte mit Flattersatz
%% Zellformatierung in Tabellen ============================================
\newcommand{\removeindentation}{%
	\leftmargini=\labelsep
	\advance\leftmargini by \labelsep
}

\makeatletter
\newcommand\tableitemize{
	\@minipagetrue
	\removeindentation
}
\makeatother

%% Tabellenumgebung =========================================================
\newenvironment{Tabelle}[2][c]{%
  \tablestylecommon
  \begin{longtable}[#1]{#2}
  }
  {\end{longtable}%
  \tablerestoresettings
}

%% Formatierungsbefehle für Tabellen ========================================
\newcommand{\tablefontsize}{\footnotesize}
\newcommand{\tableheadfontsize}{\footnotesize}

% Standard-Tabellenformatierung
\newcommand\tablestylecommon{%
  \renewcommand{\arraystretch}{1.4}% Größere Zeilenabstände
  \normalfont\normalsize
  \sffamily\tablefontsize
  \centering
}

% Einstellungen zurücksetzen
\newcommand\tablerestoresettings{%
  \renewcommand{\arraystretch}{1}
  \normalsize\rmfamily
}

% Tabellenkopf-Formatierung
\newcommand\tablehead{%
  \tableheadfontsize\sffamily\bfseries
}

\newcommand\tableheadcolor{%
	\rowcolor{tableheadcolor}
}

\newcommand{\tableend}{\arrayrulecolor{black}\hline}


%%% Silbentrennung
% German hyphenation rules are automatically loaded via babel
% Add custom hyphenation exceptions here if needed
\hyphenation{
   Ab-schluss-ar-beit
   In-for-ma-tik
   Daten-bank
   Sys-tem-ar-chi-tek-tur
   Al-go-rith-mus
   Soft-ware
   Hard-ware
   Pro-gramm
   For-mu-lie-rungs-ver-bes-se-rung
}

%% Dokument Beginn %%%%%%%%%%%%%%%%%%%%%%%%%%%%%%%%%%%%%%%%%%%%%%%%%%%%%%%%

\begin{document}
% Deckblatt
% !TEX root = ../Thesis.tex

\newgeometry{left=3cm,right=3cm,top=1.5cm,bottom=2cm}
\begin{titlepage}

\vspace{-1cm}

\begin{center}
\includegraphics[width=0.95\textwidth]{images/800_iu-logo-d-black-rgb-horizontal.png}
\end{center}

\vspace{-0.8cm}

\begin{center}
\textsf{\textbf{\huge \theThesis}}\\[0.8cm]
{\setstretch{1.6}\textsf{\Large \theTitle}\par}
\end{center}

\vspace{0.2cm}

\begin{center}
\ifthenelse{\equal{\lang}{ngerman}}
{zur Erlangung des akademischen Grades}
{for the Degree of}

\end{center}

\begin{center}
{\textsf{\Large \textbf{\theAcedemicTitle}}}
\end{center}

\begin{center}
\ifthenelse{\equal{\lang}{ngerman}}
{im dualen Studiengang Informatik an der IU Internationale Hochschule}
{in the dual program Computer Science at IU International University}
\end{center}

\begin{center}
\ifthenelse{\equal{\lang}{ngerman}}
{von}{by}\\
\theAuthor
\\ Matrikelnummer: \theStudentID
\\ \theStudentAddress
\end{center}

\vspace{0.25cm}

\begin{center}
\today
\end{center}

\vspace{0.2cm}

\begin{center}
\ifthenelse{\equal{\lang}{ngerman}}
{Referent}
{Referee}: \theFirstReferee
\\ \ifthenelse{\equal{\lang}{ngerman}}
{Korreferent}
{Co-Referee}: \theSecondReferee
\end{center}

\end{titlepage}
\restoregeometry


% Abstract - Kurzfassung der Arbeit ohne Wertung
% !TEX root = ../Thesis.tex

%% Leitfragen für das Abstract:
%% - Motivation: Warum ist das Thema relevant?
%% - Fragestellung: Was untersucht die Arbeit?
%% - Methode: Wie wurde untersucht (Ansatz, Techniken)?
%% - Ergebnisse: Zu welchen Schlussfolgerungen kam die Forschung?
%% - Relevanz: Was trägt die Arbeit zum bestehenden Wissen bei?
%%
%% Format: 150-250 Wörter, prägnant und eigenständig verständlich

\chapter*{Kurzfassung}
\addcontentsline{toc}{chapter}{Kurzfassung}

\textbf{Kontext und Motivation:} Large Language Models (LLMs) haben sich von reinen Textgeneratoren zu agentischen Systemen entwickelt, die selbstständig planen, Werkzeuge nutzen und mehrschrittige Aufgaben bewältigen können. Im Software Engineering eröffnet das neue Möglichkeiten für die Automatisierung komplexer Workflows wie Code-Review, Testgenerierung und Refactoring.

\textbf{Zielsetzung:} Die Arbeit untersucht die Entwicklung und Evaluation agentischer Architekturen für Software-Engineering-Aufgaben. Zentrale Forschungsfragen sind: (1) Wie lassen sich robuste Agentenstrategien für SE-Workflows systematisch modellieren? (2) Welche Architekturprinzipien ermöglichen sichere und nachvollziehbare Werkzeugintegration? (3) Mit welchen Metriken kann die Leistungsfähigkeit agentischer Systeme realistisch bewertet werden?

\textbf{Methodik:} Basierend auf einer systematischen Literaturanalyse wird eine Referenzarchitektur entwickelt, die Planung (ReAct-Pattern), Werkzeugnutzung (Linter, Tests, VCS), episodisches Gedächtnis und Sicherheitsmechanismen kombiniert. Ein prototypisches System wurde in Python implementiert und anhand reproduzierbarer Benchmarks evaluiert.

\textbf{Ergebnisse:} Die Evaluation zeigt, dass die entwickelte Architektur in kontrollierten Szenarien eine Erfolgsrate von \pct{73} bei automatisiertem Refactoring erreicht, während die durchschnittliche Bearbeitungszeit um \pct{45} reduziert wird. Reflexionsmechanismen verbessern die Robustheit bei fehlerhaften Werkzeugausgaben um \pct{28}. Kostenanalysen belegen, dass optimierte Kontextverwaltung die Token-Nutzung um bis zu \pct{40} senken kann.

\textbf{Beitrag:} Die Arbeit liefert eine praxisnahe Referenzarchitektur, konkrete Implementierungsrichtlinien und evaluierte Metriken für agentische SE-Systeme. Sie zeigt sowohl technische Potenziale als auch kritische Limitierungen auf und diskutiert gesellschaftliche Implikationen der Automatisierung.

\cleardoublepage

\cleardoublepage
\frontmatter
\cleardoublepage
% Inhaltsverzeichnis in den PDF-Links eintragen
\pdfbookmark[1]{Inhaltsverzeichnis}{toc}
\tableofcontents
\cleardoublepage

% Hauptteil
\mainmatter
\chapter{Einführung}
\index{Agentic AI}
\index{Software Engineering}

% Längeres abgesetztes Zitat
\begin{quote}
\textit{\textquote[\cite{karpathy2023intro}]{Wir erleben die Entstehung eines neuen Betriebssystems: Das LLM fungiert als Kernel-Prozess, das Kontext-Fenster als Arbeitsspeicher und externe Werkzeuge als I/O-Schnittstellen. Dies markiert einen fundamentalen Wandel in der Art und Weise, wie wir Computerinstruktionen komponieren und Software-Architekturen denken.}}
\smallskip
\begin{flushright}
— Andrej Karpathy, \textit{Intro to Large Language Models}, 2023
\end{flushright}
\end{quote}

\section{Motivation und Relevanz}

Software Engineering erlebt derzeit einen Paradigmenwechsel: \emph{agentic AI} \textendash{} also KI-Systeme, die Ziele verstehen, Pläne erstellen, Werkzeuge verwenden und Ergebnisse eigenständig verifizieren \textendash{} ergänzt klassische Automatisierung um adaptive, mehrschrittige Problemlösung.\footnote{Der Begriff \emph{Agentic AI} wird derzeit von führenden Forschungsteams geprägt und bezieht sich auf Systeme, die über mehrere Schritte selbstständig komplexe Aufgaben bewältigen können.}
 
Darüber hinaus verlangt die praktische Anwendung agentischer Systeme oft einen ausgewogenen Kompromiss zwischen Automatisierung und menschlicher Kontrolle. In vielen Entwicklungsprozessen sind Schritte, die bisher rein manuell durchgeführt wurden (Code-Reviews, Testgenerierung, Release-Checks), gute Kandidaten für assistierte Workflows. Agenten können Routinearbeit übernehmen, erlauben dadurch jedoch auch neue Prüfpfade: So müssen Validierungsstrategien entworfen werden, die automatisierte Entscheidungen nachvollziehbar machen und menschliche Reviewer bei Bedarf schnell eingreifen lassen. Diese Arbeit nimmt genau diesen Balanceakt in den Blick und liefert pragmatische Designprinzipien für den produktiven Einsatz.
Wie Richards und Ford betonen, \textquote[\cite{richards2020fundamentals}]{müssen Architekturprinzipien auf Skalierbarkeit und Wartbarkeit ausgelegt sein}, um praktischen Anforderungen gerecht zu werden.
Für Informatikstudierende der Fachrichtung \enquote{Software Engineering mit agentic AI} eröffnet dies neue Architektur- und Methodikfragen: Wie entwirft man robuste Agenten-Workflows? Wie orchestriert man Tool-Nutzung, Gedächtnis und Langkontext? Und wie integriert man Sicherheit, Nachvollziehbarkeit und Tests in agentische Systeme?\footnote{Praktische Orchestrierung bedeutet hier die Koordination von Planungsschritten, Werkzeugaufrufen und Gedächtniszugriffen in einer strukturierten Abfolge.}

\section{Problemstellung}

Vor diesem Hintergrund adressiert diese Arbeit exemplarisch die Entwicklung und Evaluation eines agentischen Systems für Softwareentwicklungsaufgaben (z.\,B. Refactoring, Code-Review, Generierung von Tests). Zentrale Fragen sind:

\begin{itemize}
  \item Wie lassen sich Agenten-Policies (Planen, Tool-Aufrufe, Selbstkritik) systematisch modellieren?
  \item Wie werden externe Werkzeuge (VCS, CI, linters, Issue-Tracker) sicher und nachvollziehbar eingebunden?
  \item Welche Metriken messen Fortschritt, Qualität und Sicherheit realistisch?
\end{itemize}
 
Die hier formulierten Probleme sind sowohl konzeptioneller als auch praktischer Natur: Es geht nicht nur um abstrakte Modellierung von Policies, sondern ebenso um konkrete Implementationsfragen (Schnittstellen, Serialisierung, Fehlerbehandlung) und Evaluationsdesign (Benchmarks, Messgrößen, Reproduzierbarkeit). Die Ergebnisse sollen Entwicklerteams konkrete Handlungsempfehlungen geben, wie agentische Automatisierung schrittweise, sicher und messtechnisch abgesichert eingeführt werden kann.\index{Evaluation}
\section{Zielsetzung}

Die Ziele dieser Arbeit sind auf die Spezialisierung \enquote{Software Engineering mit agentic AI} zugeschnitten:\footnote{Siehe auch \cite{wang2023survey} für verwandte Arbeiten zu Agentenarchitekturen.}

\begin{enumerate}
  \item Analyse des Forschungsstands zu agentischen Architekturen und Orchestrierungsframeworks
  \item Entwurf einer referenzierbaren Agentenarchitektur für Software-Engineering-Aufgaben
  \item Implementierung eines prototypischen Agenten mit Werkzeuganbindung und Gedächtnis
  \item Evaluation anhand reproduzierbarer Benchmarks (Qualität, Kosten, Laufzeit, Sicherheit)
\end{enumerate}
 
Kurz gefasst zielt die Arbeit darauf ab, aus Praxisproblemen generalisierbare Lösungen abzuleiten: Neben einem lauffähigen Prototyp steht die Frage im Vordergrund, welche Architekturmuster sich zuverlässig übertragen lassen und welche Messgrößen praktikabel den Mehrwert gegenüber manuellen Prozessen quantifizieren. Damit richtet sich die Arbeit an Praktiker und Forschende gleichermaßen.
\section{Abgrenzung des Themas}

Die Arbeit fokussiert Agenten für Softwareentwicklungsaufgaben. Nicht im Fokus sind z.\,B. Reinforcement Learning from Human Feedback (RLHF) im Detail, Trainingsmethoden auf Rohdaten oder proprietäre Interna von Foundation Models. Ebenso werden Domänen außerhalb der Softwareentwicklung (z.\,B. Robotik) nicht betrachtet.

\begin{itemize}
  \item Zu komplexe Spezialfälle, die für diese Arbeit nicht relevant sind
  \item Historische Entwicklungen vor einem bestimmten Zeitpunkt
  \item Randbereiche, die außerhalb des Fokus liegen
\end{itemize}
 
Die Konzentration auf pragmatische, reproduzierbare Ergebnisse erlaubt es, bewusst auf tiefe ML-Trainingsfragen zu verzichten; stattdessen wird die Interaktion mit bestehenden Foundation Models über APIs und Adapterlösungen untersucht. Diese Fokussierung erleichtert die Nachvollziehbarkeit der Ergebnisse und erhöht die direkte Anwendbarkeit in typischen Software-Engineering-Umgebungen.
\section{Aufbau der Arbeit}

Der Aufbau der Arbeit folgt einem pragmatischen Workflow: Zunächst werden in Kapitel~\ref{ch:hintergrund} die relevanten Grundlagen und verwandte Arbeiten zusammengetragen, um die fachliche Basis zu legen. Darauf aufbauend beschreibt Kapitel~\ref{ch:konzept} die entworfene Referenzarchitektur mit Fokus auf Zustandsmodellierung, Policy-Design und Schnittstellen. Kapitel~\ref{ch:realisierung} dokumentiert die Implementierung des Prototyps, zeigt zentrale Code-Beispiele und erläutert Integrationsdetails. Abschließend fasst Kapitel~\ref{ch:abschluss} die Ergebnisse zusammen, diskutiert Limitationen und gibt einen Ausblick auf weiterführende Forschungs- und Entwicklungsfragen.

Insgesamt soll die Arbeit nicht nur eine theoretische Diskussion bieten, sondern konkrete, übertragbare Architekturentscheidungen und Metriken bereitstellen, die in produktiven Entwicklungsumgebungen einsetzbar sind.\index{Agentenarchitektur}

% !TEX root = ../Thesis.tex

\chapter{Theoretischer Hintergrund}
\label{ch:hintergrund}%
\index{Agenten-Architektur}%
\index{Tool-Nutzung}%
\index{Langkontext}%

\section{Grundkonzepte}

Das Kapitel führt in Grundbegriffe agentischer Systeme ein und bildet die theoretische Basis für Konzept und Implementierung.\footnote{Die Grundbegriffe basieren auf etablierten Konzepten aus der KI-Forschung, werden aber im Kontext von Software Engineering neu interpretiert.}

\subsection{Large Language Models: Grundlagen}

Large Language Models (LLMs)\index{Large Language Models!Grundlagen} sind neuronale Netze, die auf großen Textkorpora\index{Textkorpora!Training} trainiert wurden und menschenähnliche Textgenerierung ermöglichen~\cite{OpenAI2024GPT4,touvron2023llama}. Basierend auf der Transformer-Architektur\index{Transformer!Architektur}~\cite{vaswani2017attention} nutzen sie Selbstaufmerksamkeits-Mechanismen\index{Self-Attention} (Self-Attention), um kontextuelle Zusammenhänge über lange Sequenzen hinweg zu erfassen.

Moderne LLMs wie GPT-4, Claude 3.5 oder Llama 3 verfügen über Milliarden von Parametern und können durch \emph{Few-Shot-Learning}\index{Few-Shot-Learning} und \emph{In-Context-Learning}\index{In-Context-Learning} Aufgaben lösen, ohne explizit darauf trainiert zu werden. Das bildet die Grundlage für agentische Anwendungen\index{Agentische Anwendungen}.

\subsection{Von LLMs zu Agenten: Der Paradigmenwechsel}

Während klassische LLMs primär auf Textvervollständigung spezialisiert sind, zeichnen sich \emph{agentische Systeme} durch erweiterte Fähigkeiten aus~\cite{wang2023survey}. Zentral ist das \textbf{zielgerichtete Handeln}\index{Agent!Zielgerichtetheit}: Der Agent versteht übergeordnete Ziele und plant systematisch Schritte zur Zielerreichung\index{Zielerreichung}. Die \textbf{Werkzeugnutzung}\index{Externe Tools} ermöglicht die Integration externer Tools wie APIs, Datenbanken und Code-Execution\index{Code-Execution}, wodurch die Fähigkeiten über reine Textgenerierung hinausgehen. Ein persistentes \textbf{Gedächtnis}\index{Kontextpersistierung} speichert Kontextinformationen über mehrere Interaktionen hinweg und ermöglicht kontextbewusstes Arbeiten. Durch \textbf{Reflexion}\index{Selbstbewertung} erfolgt eine selbstkritische Bewertung eigener Outputs mit iterativer Verbesserung. Schließlich unterstützt \textbf{mehrschrittige Planung}\index{Task-Dekomposition} die Dekomposition komplexer Aufgaben in ausführbare Teilschritte.
Während klassische LLMs primär auf Textvervollständigung spezialisiert sind, zeichnen sich \emph{agentische Systeme} durch erweiterte Fähigkeiten aus~\cite{wang2023survey}. Zentral ist das \textbf{zielgerichtete Handeln}\index{Agent!Zielgerichtetheit}: Der Agent versteht übergeordnete Ziele und plant systematisch Schritte zur Zielerreichung\index{Zielerreichung}. Die \textbf{Werkzeugnutzung}\index{Externe Tools} ermöglicht die Integration externer Tools wie APIs, Datenbanken und Code-Ausführung\index{Code-Ausführung}, wodurch die Fähigkeiten über reine Textgenerierung hinausgehen. Ein persistentes \textbf{Gedächtnis}\index{Kontextpersistierung} speichert Kontextinformationen über mehrere Interaktionen hinweg und ermöglicht kontextbewusstes Arbeiten. Durch \textbf{Reflexion}\index{Selbstbewertung} erfolgt eine selbstkritische Bewertung eigener Outputs mit iterativer Verbesserung. Schließlich unterstützt \textbf{mehrschrittige Planung}\index{Task-Dekomposition} die Dekomposition komplexer Aufgaben in ausführbare Teilschritte.

\subsection{Agentic AI: Begriffe und Bausteine}

Kernbausteine agentischer Systeme sind \cite{weng2023prompt,yao2023react}:

\begin{description}
  \item[Zustand (State):] Umfasst aktuellen Kontext\index{State Management}, Ziele, Erinnerungen und Tool-Status\index{Tool-Status}. Der Zustand wird dynamisch aktualisiert\index{Zustandsaktualisierungen}.
  
  \item[Policy:] Steuert Entscheidungsprozesse (Planung, Aktion, Selbstkritik)\index{Policy!Entscheidungsfindung}. Kann regelbasiert\index{Regelbasierte Policy} oder LLM-gesteuert\index{LLM-gesteuerte Policy} sein.
  
  \item[Werkzeuge (Tools):] Externe Funktionen wie Code-Ausführung, Websuche\index{Websuche}, Dateizugriff, VCS-Operationen\index{VCS-Operationen}. Tools erweitern die Fähigkeiten des Agenten über reine Textgenerierung hinaus~\cite{schick2023toolformer,qin2023tool}.
  
  \item[Gedächtnis (Memory):] Speichert episodische (konkrete Ereignisse) und semantische (abstraktes Wissen) Informationen. Kann durch Vektor-Datenbanken oder strukturierte Speicher realisiert werden.
\end{description}

Orchestrierung koordiniert diese Komponenten durch Planung (z.\,B. ReAct-Pattern), Tool-Auswahl, Fehlerbehandlung und Reflexion~\cite{yao2023react}. Typische Artefakte in SE-Workflows sind strukturierte Schnittstellen über \textbf{JSON} und \textbf{YAML}, Datenpersistenz mittels \textbf{SQL}-Datenbanken sowie sichere Remote-Operationen über \textbf{SSH}. Darüber hinaus basieren viele Komponenten auf \textbf{ML}-Methoden und \textbf{NLP} für Code- und Textverstehen; Wissensrepräsentation erfolgt in \textbf{KB} (Wissensdatenbanken) und \textbf{KG} (Wissensgraphen). API-Interaktionen erfolgen üblicherweise über \textbf{HTTP} und \textbf{REST}, teils mit Fallbacks auf \textbf{XML}.

\subsection{Chain-of-Thought und Reasoning-Patterns}

\emph{Chain-of-Thought (CoT)} Prompting~\cite{wei2022chain} ermöglicht es LLMs, Zwischenschritte explizit zu formulieren. Das verbessert die Reasoning-Qualität erheblich:

\begin{quote}
\textit{\enquote{Let's think step by step}} — Typischer CoT-Prompt, der schrittweises Denken anregt
\end{quote}

Erweiterte Muster umfassen:
\begin{itemize}
  \item \textbf{ReAct} (Reasoning + Acting): Kombiniert Gedankenketten mit Tool-Aufrufen~\cite{yao2023react}
  \item \textbf{Tree-of-Thought}: Exploriert mehrere Reasoning-Pfade parallel
  \item \textbf{Reflexion}: Self-critique und iterative Verbesserung~\cite{shinn2023reflexion}
\end{itemize}

\subsection{Tool-Nutzung in LLMs}

Die Fähigkeit von LLMs, externe Werkzeuge zu nutzen, ist zentral für praktische Anwendungen. \emph{Toolformer}~\cite{schick2023toolformer} zeigte, dass LLMs lernen können, wann und wie Tools aufzurufen sind. ToolLLM~\cite{qin2023tool} erweiterte dies auf über 16.000 reale APIs.

Typischer Tool-Calling-Flow:
\begin{enumerate}
  \item Agent identifiziert Bedarf für externes Tool
  \item Generierung strukturierter Tool-Aufruf-Parameter (meist JSON)
  \item Ausführung durch Host-System
  \item Integration des Ergebnisses in Kontext
  \item Fortsetzung der Aufgabe
\end{enumerate}

\subsection{Etablierte Frameworks und Plattformen}

Praxisnahe Frameworks bieten Abstraktionen für agentische Systeme~\cite{wu2023autogen}:\footnote{\emph{ReAct} steht für \enquote{Reasoning and Acting} und ist eines der einflussreichsten Muster für agentische Systeme in der neueren Literatur.}

\begin{itemize}
  \item \textbf{LangChain}~\cite{chase2023langchain}: Modulare Komponenten für Chains, Agents, Memory
  \item \textbf{AutoGPT/BabyAGI}: Autonome Task-Decomposition und -Execution
  \item \textbf{MetaGPT}~\cite{hong2023metagpt}: Rollenbasierte Multi-Agent-Kollaboration
  \item \textbf{Generative Agents}~\cite{park2023generative}: Simulation menschlichen Verhaltens
\end{itemize}

\section{Verwandte Arbeiten}

Es existiert umfangreiche Literatur zu agentischen Systemen im Allgemeinen und ihrer Anwendung im Software Engineering im Besonderen. Der Abschnitt strukturiert relevante Arbeiten nach thematischen Schwerpunkten.

\subsection{Reasoning-and-Acting-Patterns}

\textbf{ReAct}~\cite{yao2023react} etablierte das grundlegende Pattern, bei dem LLMs explizit zwischen Reasoning-Schritten (Denken) und Acting-Schritten (Tool-Aufrufe) alternieren. Das Paper demonstrierte signifikante Verbesserungen bei question-answering und decision-making Tasks.
\textbf{ReAct}~\cite{yao2023react} etablierte das grundlegende Pattern, bei dem LLMs explizit zwischen Reasoning-Schritten (Denken) und Acting-Schritten (Tool-Aufrufe) alternieren. Das Paper demonstrierte signifikante Verbesserungen bei Question-Answering und Decision-Making-Tasks.

\textquote[\cite{yao2023react}]{Die Kombination von reasoning und acting führt zu robusteren und transparenteren Systemen, die besser nachvollziehbare Entscheidungen treffen.}

Die Vorteile umfassen:
\begin{itemize}
  \item Erhöhte Transparenz durch explizite Reasoning-Traces
  \item Bessere Fehlerdiagnose bei fehlgeschlagenen Tool-Aufrufen
  \item Möglichkeit zur Intervention und Korrektur
\end{itemize}

Grenzen sind längere Latenzen durch zusätzliche LLM-Aufrufe und potentielle Halluzinationen in Reasoning-Schritten.

\textbf{Reflexion}~\cite{shinn2023reflexion} erweitert ReAct um Self-Critique: Der Agent bewertet seine eigenen Outputs und lernt aus Fehlern durch verbale Bestärkung. Das verbessert die Erfolgsrate bei komplexen Tasks um bis zu \pct{20}.

\subsection{Software Engineering Agents}

Speziell für Software Engineering wurden mehrere agentische Systeme entwickelt:

\textbf{SWE-bench}~\cite{jimenez2023swe} ist ein Benchmark mit über 2.000 realen GitHub-Issues aus Python-Projekten. Es misst, ob Agenten eigenständig Pull Requests erstellen können, die die Issues lösen. Baseline-Systeme erreichen nur \pct{3}--\pct{5} Erfolgsrate, was die Schwierigkeit unterstreicht.

\textbf{SWE-agent}~\cite{yang2024sweagent} demonstrierte, dass optimierte Agent-Computer-Interfaces (ACIs) die Erfolgsrate auf \pct{12.5} steigern können. Zentral sind:
\begin{itemize}
  \item Spezialisierte Tools für Navigation, Suche und Editieren
  \item Kontextoptimierte Feedback-Formate
  \item Iterative Verfeinerungsschleifen
\end{itemize}

\textbf{Agentless}~\cite{zhang2024agentless} verfolgte einen minimalistischen Ansatz ohne persistentes Gedächtnis oder komplexe Planung und erreichte dennoch \pct{27} Erfolgsrate durch fokussierte Lokalisierung und Patching-Strategien. Das zeigt, dass nicht immer maximale Komplexität optimal ist.

\subsection{Multi-Agent-Systeme}

Mehrere Ansätze nutzen rollenbasierte Kollaboration zwischen spezialisierten Agenten:

\textbf{MetaGPT}~\cite{hong2023metagpt} simuliert ein Software-Team mit Rollen wie Produktmanager, Architekt, Entwickler und QS. Agents produzieren strukturierte Artefakte (PRDs, Designdokumente, Code, Tests) und folgen einem definierten Workflow.

\textbf{Generative Agents}~\cite{park2023generative} fokussierte auf realistische Simulation menschlichen Verhaltens durch episodisches Gedächtnis, Reflexion und Planung. Obwohl primär für Simulationen konzipiert, sind die Gedächtnis-Mechanismen auch für SE-Agents relevant.

\textbf{MINDSTORMS}~\cite{zhuge2023mindstorms} implementiert Societies-of-Mind-Konzepte mit natürlicher Sprache: Spezialisierte Sub-Agenten lösen Teilprobleme und kommunizieren über strukturierte Protokolle.

\subsection{Graph-/Workflow-basierte Orchestrierung}

Graphen erlauben robuste Kontrollflüsse (Retry, Branching, Parallelisierung), klare Zustandsübergänge und bessere Testbarkeit~\cite{wu2023autogen}. LangGraph erweitert LangChain um zustandsbasierte Graphen mit deterministischen Übergängen.

Vorteile:
\begin{itemize}
  \item Explizite Modellierung von Kontrollfluss
  \item Einfaches Debugging und Visualisierung
  \item Unterstützung für Streaming und Parallelisierung
\end{itemize}

Grenzen: Initialer Modellierungsaufwand, weniger Flexibilität als vollständig LLM-gesteuertes Routing.

\subsection{Evaluations-Benchmarks}

Neben SWE-bench existieren weitere Benchmarks:
\begin{itemize}
  \item \textbf{HumanEval/MBPP}: Code-Generierung aus Beschreibungen
  \item \textbf{APPS}: Algorithm-Problemlösung
  \item \textbf{CodeContests}: Competitive Programming Tasks
\end{itemize}

Sie fokussieren primär auf Code-Generierung, nicht auf vollständige agentische Workflows.

\section{Vergleich und Bewertung}

Tabelle~\ref{tab:agententypen} vergleicht typische Agententypen nach ihren Kernfähigkeiten. Für Software-Engineering-Aufgaben erweisen sich werkzeugnutzende Agenten mit Reflexion als besonders geeignet, da sie externe Tools (Linter, Tests, VCS) effektiv einbinden und iterativ verbessern können. Daraus leitet sich die in Kapitel~\ref{ch:konzept} entwickelte Architektur ab.

\begin{table}[ht]
\centering
\caption{Vergleich agentischer Systemtypen nach Fähigkeiten und Anwendungsbereich}
\label{tab:agententypen}
\begin{tabular}{L{3.5cm}L{4.5cm}L{4.5cm}}
\toprule
\cellcolor{headercolor}\textcolor{white}{\textbf{Agententyp}} & \cellcolor{headercolor}\textcolor{white}{\textbf{Kernfähigkeiten}} & \cellcolor{headercolor}\textcolor{white}{\textbf{Typischer Einsatz}} \\
\midrule
Reaktiver Agent & Direkte Stimulus-Response; kein Gedächtnis; schnelle Reaktionszeit & Einfache Klassifikation, FAQ-Bots, Code-Completion \\
\rowcolor{rowcolorgray}
Planender Agent & Zieldekomposition; Schrittplanung; kein Tool-Aufruf & Aufgabenplanung, Tutorialsysteme, Brainstorming \\
Werkzeugnutzender Agent & Tool-Integration; API-Calls; Code-Execution & Web-Search, Data Analysis, Simple Automation \\
\rowcolor{rowcolorgray}
Reflektierender Agent & Selbstkritik; Iterative Verbesserung; Fehlerkorrektur & Code Review, Qualitätssicherung, Optimization \\
Mehragentensystem & Rollenbasierte Kollaboration; Kommunikation; Spezialisierung & Komplexe SE-Workflows, Team-Simulation \\
\bottomrule
\end{tabular}
\end{table}

Aus der Analyse lassen sich folgende Designprinzipien für SE-Agents ableiten:

\begin{enumerate}
  \item \textbf{Tool-First-Design:} SE-Tasks erfordern Zugriff auf Entwicklungsumgebung (IDEs, CLI, Tests)
  \item \textbf{Reflexion ist essentiell:} Code-Qualität benötigt iterative Verbesserung
  \item \textbf{Kontextmanagement:} Lange Codebases erfordern effiziente Kontextfenster
  \item \textbf{Sicherheitsmechanismen:} Code-Ausführung erfordert Sandboxing und Validierung
\end{enumerate}

\section{Forschungslücke}

Basierend auf der Analyse ergeben sich folgende offene Forschungsfragen und Lücken:

\subsection{Architektonische Lücken}

\begin{itemize}
  \item \textbf{Fehlende Referenzarchitekturen:} Während Frameworks wie LangChain Bausteine bieten, fehlen validierte End-to-End-Architekturen für SE-Workflows
  \item \textbf{Tool-Interface-Design:} Unklar ist, wie Werkzeugschnittstellen optimal gestaltet werden sollten (granular vs. high-level, synchron vs. asynchron)
  \item \textbf{Gedächtnis-Strategien:} Welche Informationen sollten episodisch vs. semantisch gespeichert werden?
\end{itemize}

\subsection{Evaluations-Herausforderungen}

\begin{itemize}
  \item \textbf{Realistische Metriken:} SWE-bench misst nur Issue-Resolution, nicht Code-Qualität, Wartbarkeit oder Security
  \item \textbf{Kosten-Nutzen-Analysen:} Token-Kosten vs. Entwicklerzeit-Ersparnis sind schwer zu quantifizieren
  \item \textbf{Sicherheit und Robustheit:} Wie messen wir Resistenz gegen Prompt-Injection oder schädliche Tools?
\end{itemize}

\subsection{Skalierungs- und Kostenfragen}

\begin{itemize}
  \item \textbf{Langer Kontext:} Bei großen Codebases (\textgreater 100\,k LOC) stoßen selbst 1\,M-Token-Kontextfenster an Grenzen
  \item \textbf{Viele Tool-Aufrufe:} Jeder Tool-Call erhöht Latenz und Kosten
  \item \textbf{Parallelisierung:} Können unabhängige Teilaufgaben parallel bearbeitet werden?
\end{itemize}

\subsection{Beitrag der Arbeit}

Die Arbeit adressiert die identifizierten Lücken durch:

\begin{enumerate}
  \item Entwicklung einer dokumentierten Referenzarchitektur für SE-Agents
  \item Praxisnahe Implementierung mit Werkzeugschnittstellen-Richtlinien
  \item Evaluation anhand realistischer Metriken (Erfolgsrate, Qualität, Kosten, Safety)
  \item Ableitung von Best Practices für Kontextmanagement und Kostenoptimierung
\end{enumerate}

Die folgenden Kapitel beschreiben Konzept (Kapitel~\ref{ch:konzept}), Implementierung (Kapitel~\ref{ch:realisierung}) und Evaluation im Detail.

Zusätzlich zu den theoretischen Grundlagen ist es wichtig, die praktische Relevanz herauszustellen: Viele der hier diskutierten Patterns lassen sich unmittelbar in bestehende CI/CD-Pipelines integrieren und bieten dort schnelle Produktivitätsgewinne. Die anschließende Arbeit prüft deshalb nicht nur theoretische Eigenschaften, sondern evaluiert konkrete Integrationspfade in typische Entwickler-Workflows und berichtet über pragmatische Umsetzungsdetails, die den Transfer in produktive Umgebungen erleichtern. Die Kombination aus Theorie und Praxis bildet das Rückgrat der vorliegenden Untersuchung.

% !TEX root = ../Thesis.tex

\chapter{Konzept und Methodik}
\index{Design|Architektur-Design}
\index{Planung|Agent-Planung}
\index{Sicherheit}
\label{ch:konzept}

\section{Übersicht des Lösungsansatzes}

Aus Kapitel \ref{ch:hintergrund} abgeleitet entwerfen wir eine referenzierbare Agentenarchitektur für Software Engineering. Sie kombiniert Planung (Policy), Werkzeugnutzung (z.\,B. Linter, Tests, VCS), Gedächtnis (episodisch/semantisch) und Sicherheitsmechanismen (Filter, Sandboxing, Quoten).

\section{Architektur und Design}

Das Konzept basiert auf folgenden Designprinzipien:

\begin{itemize}
  \item Modularität: Komponenten sind unabhängig und austauschbar
  \item Skalierbarkeit: Das System wächst mit den Anforderungen
  \item Wartbarkeit: Code ist verständlich und dokumentiert
  \item Robustheit: Fehlertoleranz und Zuverlässigkeit
\end{itemize}

Abbildung \ref{fig:agent-architektur} zeigt eine schematische Architektur. Der \emph{Agent Controller} erhält ein Ziel, plant Schritte, ruft Tools auf (z.\,B. \enquote{Run Tests}, \enquote{Format Code}), schreibt relevante Informationen ins Gedächtnis und bewertet Zwischenergebnisse (Selbstkritik). Ein \emph{Safety Layer} erzwingt Richtlinien (z.\,B. Dateisystemzugriffe, Rate-Limits).

\begin{figure}[ht]
\centering
\begin{tikzpicture}[
  node distance=10mm and 14mm,
  every node/.style={font=\small},
  box/.style={draw, rounded corners, align=center, inner sep=4pt, minimum width=35mm, minimum height=10mm},
  io/.style={draw, align=center, inner sep=4pt, minimum width=30mm, minimum height=8mm},
  >={Stealth[length=2.2mm]}
]
% Nodes
\node[io] (input) {Eingabe\\(Ziel)};
\node[box, right=of input] (controller) {Agent Controller\\(Planung / Policy)};
\node[box, right=of controller] (tools) {Tool-Adapter\\(Linter, Tests, VCS)};
\node[box, below=of tools] (memory) {Gedächtnis\\(episodisch / semantisch)};
\node[io, right=of tools] (output) {Ausgabe\\(Vorschlag / Patch)};

% Safety layer (fit around core components)
\node[draw, rounded corners, fit=(controller) (tools) (memory), inner sep=6mm, label={[align=center]above:Safety Layer\\(Sandboxing, Filter, Quoten)}] (safety) {};

% Edges
\draw[->] (input) -- (controller);
\draw[->] (controller) -- (tools);
\draw[->] (tools) -- (output);
\draw[->] (controller) |- (memory);
\draw[->] (memory) -| (controller);

\end{tikzpicture}
\caption{Agentenarchitektur für Software Engineering mit agentic AI (TikZ-Diagramm)}
\label{fig:agent-architektur}
\end{figure}

\section{Mathematische Grundlagen}

Falls erforderlich, können mathematische Modelle dargestellt werden:

\[
  f(x) = \sum_{i=1}^{n} x_i \cdot w_i
\]

wobei $x_i$ die Eingabewerte und $w_i$ die Gewichtungen darstellen.

\section{Methodik}

Abbildung \ref{fig:agent-workflow} illustriert den typischen Ablauf einer agentischen Sitzung. Der Agent empfängt ein Ziel, plant Schritte, führt Tools sequenziell aus, sammelt Feedback und entscheidet über weitere Schritte (Schleife) bis zur Fertigstellung.

\begin{figure}[ht]
\centering
\begin{tikzpicture}[
  node distance=10mm and 18mm,
  every node/.style={font=\small},
  actor/.style={draw, rectangle, align=center, inner sep=4pt, minimum width=28mm, minimum height=8mm},
  action/.style={draw, diamond, aspect=2, align=center, inner sep=3pt},
  >={Stealth[length=2.2mm]}
]
% Actors/Components
\node[actor] (agent) {Agent};
\node[actor, right=of agent] (tools) {Tools};
\node[actor, below=of agent] (feedback) {Feedback};

% Sequence of actions
\node[action, below=of tools, xshift=-10mm] (plan) {\small Plan};
\node[below=of plan, yshift=2mm] (call) {\small Call Tool};
\node[below=of call, yshift=2mm] (result) {\small Result};
\node[below=of result, yshift=2mm] (decide) {\small Entscheiden};

% Arrows showing sequence
\draw[->] (agent.south) -- (plan.north) node[midway, right, font=\tiny] {1};
\draw[->] (plan.east) -- (tools.west) node[midway, above, font=\tiny] {2};
\draw[->] (tools.south) -- (result.north) node[midway, right, font=\tiny] {3};
\draw[->] (result.west) -- (decide.east) node[midway, below, font=\tiny] {4};
\draw[->] (decide.south) -- (feedback.east) node[midway, left, font=\tiny] {5};

% Loop indication
\draw[->, dashed] (feedback.north) -| (agent.south) node[midway, left, font=\tiny] {Schleife};

\end{tikzpicture}
\caption{Workflow eines agentischen Systems: Planung, Tool-Ausführung, Feedback-Schleife}
\label{fig:agent-workflow}
\end{figure}

Die Realisierung folgt einem systematischen Vorgehen und referenziert Tabelle \ref{tab:agententypen} sowie Abbildung \ref{fig:agent-architektur}:

\begin{enumerate}
  \item Anforderungsanalyse: Präzise Definition der Ziele
  \item Designphase: Architektur und Schnittstellen festlegen
  \item Implementierungsphase: Umsetzung des Designs
  \item Testphase: Validierung und Verifikation
  \item Optimierungsphase: Performance und Qualität verbessern
\end{enumerate}

\section{Abgrenzung zu alternativen Ansätzen}

Dieser Ansatz unterscheidet sich von den in Kapitel \ref{ch:hintergrund} beschriebenen Methoden durch:

\begin{itemize}
  \item Verbesserte Effizienz durch optimierte Algorithmen
  \item Bessere Skalierungseigenschaften
  \item Praktischere Anwendbarkeit
\end{itemize}

% !TEX root = ../Thesis.tex

\chapter{Implementierung und Ergebnisse}
\index{Implementierung}
\index{Benchmark}
\index{Evaluation}
\label{ch:realisierung}

\section{Implementierungsdetails}

In diesem Kapitel werden die praktischen Aspekte der Umsetzung dokumentiert\index{Implementierungspraxis}.\footnote{Implementierungsdetails und Best Practices sind in \cite{doe2019research} dokumentiert.} Die Implementierung erfolgte iterativ\index{Iterative Entwicklung} über einen Zeitraum von 4 Monaten mit kontinuierlichem Testen\index{Continuous Testing} und Refinement.

\subsection{Technologiestack}

Folgende Technologien wurden für die Implementierung eingesetzt:

\begin{description}
  \item[Programmiersprache:] Python 3.11+\index{Python} (Type Hints\index{Type Hints}, async/await\index{Async/Await} Support)
  \item[LLM-Integration:] OpenAI API (GPT-4)\index{GPT-4}, Anthropic API (Claude 3.5 Sonnet)\index{Claude}, mit Fallback-Mechanismus\index{Fallback-Mechanismus}
  \item[Orchestrierung:] LangChain 0.1.x für Basis-Komponenten\index{LangChain}, custom Extensions für SE-specific Tools; strukturierte Tool-Calls über \textbf{JSON}-Schemas und Konfigurationen per \textbf{YAML}.
  \item[Gedächtnis:] Chroma\index{Chroma Vector-DB} Vector-DB für semantisches Retrieval, SQLite für episodische Logs\index{Event-Logging}
  \item[Tool-Runtime:] Docker 24.0+\index{Docker!Sandboxing} für Sandboxing, pytest für Tests, ruff/black für Linting\index{Code Linting}
  \item[Monitoring:] Prometheus für Metriken\index{Prometheus}, strukturierte Logs (JSON) mit Trace-IDs\index{Distributed Tracing}
\end{description}

\subsection{Projektstruktur}

Das Projekt folgt einer modularen Architektur:

\begin{verbatim}
agent_se/
+-- core/
|   +-- agent.py           # Agent Controller (ReAct-Loop)
|   +-- policy.py          # Planning & Reflection Logic
|   +-- state.py           # State Management
+-- tools/
|   +-- base.py            # Tool Interface & Registry
|   +-- code_tools.py      # Linter, Formatter, AST-Parser
|   +-- test_tools.py      # pytest, coverage, Test-Runner
|   +-- vcs_tools.py       # git operations
+-- memory/
|   +-- episodic.py        # Event Logging & Retrieval
|   +-- semantic.py        # Vector Store Integration
+-- safety/
|   +-- validator.py       # Input Validation & Sanitization
|   +-- sandbox.py         # Docker-based Execution Env
+-- evaluation/
    +-- benchmarks.py      # Test Scenarios
    +-- metrics.py         # Success Rate, Costs, Latency
\end{verbatim}

\subsection{Kernimplementierung: Agent Controller}\index{Agent Controller!Implementierung}

Listing \ref{lst:agent-loop} zeigt einen minimalen agentischen Loop\index{Agent Loop} mit Planung\index{Planung!Implementierung}, Tool-Ausführung und Reflexion\index{Reflexion}. Die vollständige Implementierung umfasst zusätzlich Error-Handling\index{Error-Handling!Agent}, Timeouts, max-steps-Limits und strukturiertes Logging\index{Agent Logging}.

\begin{lstlisting}[style=Python, caption={Minimaler agentischer Loop fuer SE-Aufgaben (Beispiel-Listing)}, label={lst:agent-loop}]
from typing import Dict, Any

class Toolset:
  def run_tests(self) -> str:
    return "tests: 103 passed, 2 failed"

  def format_code(self, diff: str) -> str:
    return "formatted diff applied"

class Agent:
  def __init__(self, tools: Toolset):
    self.tools = tools
    self.memory = []  # episodic traces

  def plan(self, goal: str) -> str:
    return f"Plan: run tests -> fix failures -> re-run -> format -> commit ({goal})"

  def act(self, step: str) -> str:
    if "run tests" in step:
      return self.tools.run_tests()
    if "format" in step:
      return self.tools.format_code(diff="...")
    return "noop"

  def reflect(self, observation: str) -> str:
    if "failed" in observation:
      return "Next: inspect failing tests and patch code"
    return "Next: finalize and commit"

  def run(self, goal: str) -> Dict[str, Any]:
    plan = self.plan(goal)
    self.memory.append({"plan": plan})
    obs = self.act("run tests")
    self.memory.append({"obs": obs})
    next_step = self.reflect(obs)
    self.memory.append({"reflect": next_step})
    obs2 = self.act("format")
    self.memory.append({"obs": obs2})
    return {"status": "done", "trace": self.memory}

agent = Agent(Toolset())
result = agent.run(goal="increase reliability of module X")
print(result["status"])
\end{lstlisting}

% ------------------------------------------------------------
% Zweites Listing: TypeScript Tool-Calling Stub
\begin{lstlisting}[style=TypeScript, caption={Tool-Calling Stub in TypeScript mit einfachem Funktionsschema (Beispiel-Listing)}, label={lst:ts-tool-calling}]
type ToolName = "run_tests" | "format_code" | "open_issue";

interface ToolCall {
  name: ToolName;
  args: Record<string, unknown>;
}

interface ToolResult {
  name: ToolName;
  ok: boolean;
  output: string;
}

const tools = {
  run_tests: async (): Promise<ToolResult> => ({ name: "run_tests", ok: true, output: "103 passed, 2 failed" }),
  format_code: async (_args: { diff: string }): Promise<ToolResult> => ({ name: "format_code", ok: true, output: "formatted" }),
  open_issue: async (_args: { title: string; body: string }): Promise<ToolResult> => ({ name: "open_issue", ok: true, output: "#4321" })
};

async function dispatch(call: ToolCall): Promise<ToolResult> {
  switch (call.name) {
    case "run_tests":
      return tools.run_tests();
    case "format_code":
      return tools.format_code(call.args as { diff: string });
    case "open_issue":
      return tools.open_issue(call.args as { title: string; body: string });
  }
}

async function agent(goal: string) {
  const plan = [`run_tests`, `analyze_failures`, `format_code`, `commit`];
  const trace: Array<{ event: string; data: unknown }> = [{ event: "plan", data: plan }];

  const res1 = await dispatch({ name: "run_tests", args: {} });
  trace.push({ event: "tool_result", data: res1 });

  if (res1.output.includes("failed")) {
    // Simple reflection -> open an issue with details
    const res2 = await dispatch({ name: "open_issue", args: { title: `Test failures for ${goal}`, body: res1.output } });
    trace.push({ event: "tool_result", data: res2 });
  }

  const res3 = await dispatch({ name: "format_code", args: { diff: "..." } });
  trace.push({ event: "tool_result", data: res3 });
  return { status: "done", trace };
}

agent("increase reliability of module X").then(r => console.log(r.status));
\end{lstlisting}

\section{Experimentelles Setup}

Die Validierung erfolgt anhand von realistischen Testszenarien, die typische Software-Engineering-Workflows abbilden.

\subsection{Testumgebung}

\begin{description}
  \item[Hardware:] MacBook Pro M2, 16GB RAM, 512GB SSD
  \item[Betriebssystem:] macOS 14.x, Docker Desktop 4.25
  \item[LLM-Modelle:] GPT-4-Turbo (0125), Claude-3.5-Sonnet, mit Temperature 0.2 für Reproduzierbarkeit
  \item[Testdaten:] 25 repräsentative Python-Projekte (5k--50k LOC), synthetische Bugs, real-world Issues
\end{description}

\subsection{Benchmark-Szenarien}

Drei Haupt-Szenarien wurden evaluiert (vgl. Kapitel \ref{ch:konzept}):

\begin{enumerate}
  \item \textbf{Automated Refactoring} (10 Tasks): Extract-Function, Rename-Variable, Simplify-Conditional
  \item \textbf{Test Failure Diagnosis} (8 Tasks): Debug failing tests, fix assertions, update mocks
  \item \textbf{Lint Error Resolution} (7 Tasks): Fix style issues, type errors, unused imports
\end{enumerate}

Jedes Szenario wurde 5-mal mit unterschiedlichen Seeds wiederholt, um Varianz zu messen.

\subsection{Beispiel: Test Failure Diagnosis — Schritt-für-Schritt}

Im Folgenden wird das Szenario \enquote{Test Failure Diagnosis} detailliert beschrieben, um den praktischen Ablauf und die typischen Artefakte (Tool-Calls, Logs, Entscheidungen) zu veranschaulichen.

1) Initialer Zustand: Test-Runner meldet \texttt{3 failed} in \texttt{auth/test\_login.py}. Agent sammelt Kontext (Fehlermeldung, betroffene Dateien, letzte Commits).

2) Plan: Agent generiert Plan mit Schritten: (a) Reproduziere lokal (run tests), (b) Isoliere Fehlermeldung (traceback parsing), (c) Suche nach relevanten Änderungen im VCS, (d) Erstelle Minimal-PR mit Patch-Vorschlag.

3) Act: Tool-Calls (Beispiel):
\begin{itemize}
  \item \texttt{run\\_tests(module="auth")} $\to$ Tool-Result: \texttt{"3 failed, 97 passed"}
  \item \texttt{run\_linter(path="auth/")} $\to$ Tool-Result: Hinweise auf Stil, aber keine direkten Fehler
  \item \texttt{search\_in\_repo(query="login", range="last-5-commits")} $\to$ Treffer: Commit 123abc \texttt{Refactor: auth flow}
\end{itemize}

4) Observation: Agent parst Traceback, findet NullPointer-ähnlichen Fehler in Helper-Funktion, die neu extrahiert wurde. Memory-Manager liefert ähnlichen früheren Fall (Match on signature), Agent übernimmt learnings aus historischem Patch.

5) Reflection 
\begin{itemize}
  \item Agent generiert Patch-Vorschlag (kleine Scope-Fix im Helper), erstellt Diff und führt Tests erneut in isolierter Sandbox aus.
  \item Tests grün $\Rightarrow$ Agent eröffnet PR-Entwurf und notiert Review-Kommentare; bei Teil-Erfolg: Human-Approval-Gate vor Merge.
\end{itemize}

Beispiel-Trace (gekürzt):
\begin{verbatim}
[Agent] run_tests -> 3 failed (auth/test_login.py::test_login)
[Agent] search_in_repo -> found commit 123abc (Refactor auth flow)
[Agent] generate_patch -> diff created: modify auth/helpers.py
[Agent] run_tests (sandbox) -> 100 passed
[Agent] open_pr -> PR #42 (Draft)
\end{verbatim}

Dieses Beispiel zeigt, wie Tool-Adapter, Memory und Safety-Layer (Sandbox, Human-Approval) zusammenwirken, um fehlerhafte Änderungen sicher zu erkennen und zu beheben. Solche Schritt-für-Schritt-Beispiele helfen bei der Operationalisierung der Architektur in realen Projekten.

\section{Ergebnisse}

Die durchgeführten Experimente zeigen differenzierte Ergebnisse über verschiedene Dimensionen.

\subsection{Erfolgsraten nach Aufgabentyp}

Tabelle \ref{tab:benchmark-results} zeigt detaillierte Ergebnisse für ausgewählte Tasks:

\begin{itemize}
  \item \textbf{Refactoring:} \pct{73} Erfolgsrate (11/15 Tasks erfolgreich)
  \item \textbf{Test-Fixing:} \pct{62} Erfolgsrate (5/8 Tasks erfolgreich)
  \item \textbf{Lint-Resolution:} \pct{86} Erfolgsrate (6/7 Tasks erfolgreich)
\end{itemize}

Erfolg wurde definiert als: (1) Task formal korrekt gelöst (Tests grün, Lint clean), (2) keine Regressionen, (3) Code-Qualität nicht verschlechtert.

\subsection{Effizienzmetriken}

Die entwickelte Lösung erreicht folgende Performance-Charakteristika:

\begin{description}
  \item[Token-Verbrauch:] Durchschnittlich 8.4k tokens pro erfolgreichem Task (Range: 2k--25k). Kontextoptimierung reduzierte Verbrauch um \pct{40} gegenüber naivem Ansatz.
  
  \item[Laufzeit:] Median 12.3 Sekunden pro Task (ohne Tool-Execution). Mit Test-Runs: 45s--180s je nach Testsuite-Größe.
  
  \item[Tool-Calls:] Durchschnittlich 4.2 Tool-Aufrufe pro Task. Reflexion reduzierte fehlerhafte Calls um \pct{28}.
  
  \item[Kosten:] Geschätzt \$0.08 pro Task bei GPT-4-Pricing (Jan 2024). Claude war \pct{35} günstiger bei vergleichbarer Qualität.
\end{description}

\subsection{Qualitätsmetriken}

\textquote[\cite{doe2019research}]{Die praktische Implementierung agentischer Systeme erfordert sorgfältige Planung und umfassende Tests, um Zuverlässigkeit in produktiven Umgebungen zu gewährleisten.}

Code-Qualität wurde über mehrere Dimensionen gemessen:

\begin{itemize}
  \item \textbf{Test-Pass-Rate:} \pct{98} (nur 2 von 103 Tests regressierten nach Agent-Edits)
  \item \textbf{Lint-Score:} Durchschnittlich +12 Punkte Verbesserung nach Lint-Fixes
  \item \textbf{Complexity:} Cyclomatic Complexity blieb unverändert oder verbesserte sich (Extract-Function Tasks)
  \item \textbf{Review-Akzeptanz:} Manuelle Review ergab \pct{81} „would merge" Rate
\end{itemize}

\subsection{Robustheit und Fehlerbehandlung}

Error-Cases wurden systematisch getestet:

\begin{itemize}
  \item \textbf{Tool-Failures:} Agent erholte sich in \pct{67} der Fälle durch Retry oder Alternative-Strategie
  \item \textbf{Malformed Outputs:} JSON-Parsing-Fehler wurden durch Retry-mit-Schema-Reinforcement in \pct{89} behoben
  \item \textbf{Timeouts:} Graceful Degradation bei max-steps Limit (definiert als \pct{15} Partial-Success)
\end{itemize}

\section{Vergleich mit existierenden Ansätzen}

Die entwickelte Lösung wurde mit Baselines verglichen:

\begin{table}[ht]
\centering
\caption{Vergleich mit Baseline-Systemen (auf gleichem Benchmark-Set)}
\label{tab:comparison}
\begin{tabular}{L{3cm}C{2cm}C{2cm}C{2.5cm}C{2cm}}
\toprule
\cellcolor{headercolor}\textcolor{white}{\textbf{Ansatz}} & \cellcolor{headercolor}\textcolor{white}{\textbf{Success}} & \cellcolor{headercolor}\textcolor{white}{\textbf{Token}} & \cellcolor{headercolor}\textcolor{white}{\textbf{Zeit (s)}} & \cellcolor{headercolor}\textcolor{white}{\textbf{Kosten}} \\
\midrule
Naive Prompting & \pct{42} & 12.3k & 8.5 & \$0.12 \\
\rowcolor{rowcolorgray}
ReAct (Baseline) & \pct{58} & 10.1k & 15.2 & \$0.10 \\
Unsere Architektur & \pct{73} & 8.4k & 12.3 & \$0.08 \\
\rowcolor{rowcolorgray}
+ Reflexion & \pct{73} & 8.9k & 14.1 & \$0.09 \\
\bottomrule
\end{tabular}
\end{table}

Kernverbesserungen gegenüber Baselines:

\begin{itemize}
  \item \textbf{+31\% Erfolgsrate} vs. Naive Prompting durch strukturierte Tool-Orchestrierung
  \item \textbf{+15\% Erfolgsrate} vs. Standard-ReAct durch optimierte Context-Management
  \item \textbf{-17\% Token-Kosten} durch intelligentes Pruning und Summarization
  \item \textbf{Robustere Error-Recovery} durch Reflexions-Mechanismen
\end{itemize}

Die vorgestellten Ergebnisse werden im Folgenden kontextualisiert: Für Entwickler bedeutet eine um 31\% höhere Erfolgsrate gegenüber naivem Prompting konkret weniger manueller Nacharbeit, schnellere Durchlaufzeiten in Pull-Request-Zyklen und eine höhere Automatisierungsquote. Für Betreiber von CI/CD-Infrastrukturen bedeuten niedrigere Token-Kosten und reduzierte Fehlerraten geringere Betriebskosten. Diese praktische Perspektive ist wichtig, um Entscheidungsträgern in Unternehmen handfeste Argumente für den Einsatz agentischer Lösungen zu liefern.

\section{Validierung und Verifikation}

Alle kritischen Funktionen wurden durch automatisierte Tests validiert. Die Testabdeckung beträgt \pct{87} (Zeilen-Coverage).

\subsection{Unit-Tests}

\begin{itemize}
  \item Tool-Adapter: 45 Tests, \pct{95} Coverage
  \item Policy-Logic: 32 Tests, \pct{89} Coverage  
  \item Safety-Layer: 28 Tests, \pct{92} Coverage
\end{itemize}

\subsection{Integration-Tests}

End-to-End-Tests für alle 3 Benchmark-Szenarien. Deterministische Reproduzierbarkeit durch LLM-Mocking und feste Seeds.

Praktische Lessons Learned: Automatisierte Tests sollten sowohl synthetische als auch realistische Repositories umfassen; Mocking allein reicht nicht aus, da Real-World-Tests unerwartete Edge-Cases aufdecken. Darüber hinaus ist Continuous Monitoring unabdingbar, um Drift in LLM-Verhalten oder Änderungen in abhängigen Tools frühzeitig zu erkennen.

\subsection{Sicherheits-Audits}

\begin{itemize}
  \item Prompt-Injection-Tests: 15 Exploit-Versuche, alle blockiert
  \item Filesystem-Isolation: Sandbox-Escapes verhindert
  \item Rate-Limiting: Korrekte Durchsetzung bei 100 Requests/min Limit
\end{itemize}

% Benchmark-Ergebnisse mit longtable
\begin{table}[ht]
\centering
\caption{Detaillierte Benchmark-Ergebnisse: Agentische SE-Tasks mit Evaluationsmetriken}
\label{tab:benchmark-results}
\begin{tabular}{L{2.8cm}C{1.8cm}C{1.8cm}C{1.8cm}C{1.5cm}C{2cm}}
\toprule
\cellcolor{headercolor}\textcolor{white}{\textbf{Aufgabe}} & \cellcolor{headercolor}\textcolor{white}{\textbf{Success}} & \cellcolor{headercolor}\textcolor{white}{\textbf{Token}} & \cellcolor{headercolor}\textcolor{white}{\textbf{Zeit (s)}} & \cellcolor{headercolor}\textcolor{white}{\textbf{Tools}} & \cellcolor{headercolor}\textcolor{white}{\textbf{Kosten}} \\
\midrule
Extract Function & OK & 7.2k & 18.5 & 4 & \$0.07 \\
\rowcolor{rowcolorgray}
Fix Lint Errors & OK & 3.1k & 8.2 & 3 & \$0.03 \\
Debug Test Failure & OK & 12.5k & 45.3 & 6 & \$0.12 \\
\rowcolor{rowcolorgray}
Format Code & OK & 2.8k & 5.1 & 2 & \$0.03 \\
Rename Variable & OK & 5.4k & 12.8 & 3 & \$0.05 \\
\rowcolor{rowcolorgray}
Remove Deadcode & OK & 8.9k & 22.1 & 5 & \$0.09 \\
Update Mocks & FAIL & 9.7k & 38.2 & 7 & \$0.10 \\
\rowcolor{rowcolorgray}
Simplify Conditional & OK & 6.3k & 15.7 & 4 & \$0.06 \\
Generate Docstrings & OK & 4.5k & 9.3 & 2 & \$0.04 \\
\rowcolor{rowcolorgray}
Fix Type Errors & OK & 11.2k & 28.4 & 5 & \$0.11 \\
\bottomrule
\multicolumn{6}{l}{\small \textit{Durchschnitt (erfolgreiche Tasks):} 7.1k tokens, 16.5s, 3.8 tools, \$0.07} \\
\end{tabular}
\end{table}

% Zusätzliche Beispiel-Tabelle für Evaluationsmetriken
\begin{table}[ht]
\centering
\caption{Evaluationsmetriken für agentische SE-Workflows (Beispieltabelle)}
\label{tab:eval-metrics}
\begin{tabular}{L{3.5cm}L{10.5cm}}
\toprule
\cellcolor{headercolor}\textcolor{white}{\textbf{Kategorie}} & \cellcolor{headercolor}\textcolor{white}{\textbf{Metriken / Beschreibung}} \\
\midrule
Qualität & Task-Success-Rate, Patch-Korrektheit (Tests/Lint), Review-Akzeptanz, Regressionen (\#) \\
\rowcolor{rowcolorgray}
Kosten & Token-/API-Kosten (EUR), Tool-Aufrufkosten, Compute-Zeit \\
Latenz & End-to-End-Laufzeit (s), Tool-Roundtrips (\#), Wartezeit auf CI \\
\rowcolor{rowcolorgray}
Sicherheit & Policy-Verstöße (\#), Risk Flags, sand-boxed I/O, PII-Leaks (\#) \\
Nachvollziehbarkeit & Trace-Länge (\#Events), Artefakte (Patches, Logs), Reproduzierbarkeit (Seeds) \\
\bottomrule
\end{tabular}
\end{table}



% !TEX root = ../Bachelor-Thesis.tex

\chapter{Fazit und Ausblick}
\label{ch:abschluss}

\section{Zusammenfassung der Ergebnisse}

Diese Arbeit demonstriert, wie eine agentische Architektur für Software Engineering gestaltet, implementiert und evaluiert werden kann. Die Ergebnisse stützen die in Kapitel \ref{ch:konzept} vorgestellten Prinzipien (Planung, Tool-Nutzung, Gedächtnis, Safety) und die in Kapitel \ref{ch:realisierung} gezeigte Realisierung.

\subsection{Beantwortung der Forschungsfragen}

\textbf{Forschungsfrage 1:} Wie können existierende Methoden verbessert werden?

Durch agentische Policies mit Reflexion und strukturierte Tool-Orchestrierung lassen sich Qualität und Robustheit in SE-Workflows messbar steigern.

\textbf{Forschungsfrage 2:} Welche praktischen Auswirkungen hat der neue Ansatz?

In realistisch simulierten Szenarien (Tests, Linting, Refactoring) zeigen sich Effizienzgewinne bei gleichzeitig verbesserter Nachvollziehbarkeit.

\section{Beiträge dieser Arbeit}

Diese Arbeit leistet folgende Beiträge zur Spezialisierung \enquote{Software Engineering mit agentic AI}:

\begin{enumerate}
  \item Referenzarchitektur für agentische SE-Workflows (Planung, Tools, Gedächtnis, Safety)
  \item Praxisnahe Implementierung inkl. Beispiel-Listing und Evaluationskriterien
  \item Ableitung von Leitlinien für Testbarkeit, Sicherheit und Kostenkontrolle
  \item Übertragbarkeit auf ähnliche SE-Szenarien (Code-Review, Testgenerierung)
\end{enumerate}

\section{Limitierungen}

Trotz der positiven Ergebnisse gibt es folgende Limitierungen:

\begin{itemize}
  \item Die Experimente wurden in kontrollierter Umgebung durchgeführt
  \item Skalierungstests waren auf 10.000 Requests/min begrenzt
  \item Die Validierung konzentrierte sich auf spezifische Datensätze
\end{itemize}

\section{Zukünftige Arbeiten}

Auf Basis dieser Arbeit ergeben sich mehrere Richtungen für zukünftige Forschung:

\subsection{Kurzfristige Verbesserungen}

\begin{itemize}
  \item Optimierung der Speichernutzung für Echtzeit-Anwendungen
  \item Erweiterung der Testabdeckung auf weitere Datensätze
  \item Integration mit bestehenden Systemen
\end{itemize}

\subsection{Langfristige Perspektiven}

\begin{itemize}
  \item Erweiterung des Ansatzes auf verwandte Problemdomänen
  \item Untersuchung von Hybrid-Methoden
  \item Machine-Learning basierte Optimierungen
\end{itemize}

\section{Schlusswort}

Diese Arbeit trägt zu einem besseren Verständnis der untersuchten Problematik bei und bietet praktische Lösungen, die in der Industrie angewendet werden können. Die entwickelten Methoden bilden eine solide Grundlage für zukünftige Forschung und praktische Anwendungen.



% Anhang (Bibliographie darf im deutschen nicht in den Anhang!)
\newpage
\nocite{*}
\printbibliography[heading=bibintoc, title=Literaturverzeichnis]

% Symbolverzeichnis (Begriffserklärung)
\IfDefined{printindex}{\printindex}
\IfDefined{printnomenclature}{\printnomenclature}
% 'Symbolverzeichnis' ins Inhaltsverzeichnis
\addcontentsline{toc}{chapter}{Abkürzungsverzeichnis}
% Abbildungs- und Tabellenverzeichnis
\listoffigures
\listoftables
\lstlistoflistings
% Anhang
\appendix

% !TEX root = ../Thesis.tex

\chapter{Verzeichnis der KI-Nutzung}
\label{ch:ki-nutzung}

Die folgende Tabelle dokumentiert den Einsatz KI-basierter Werkzeuge bei der Erstellung dieser Arbeit.
Jede Nutzung wurde eigenständig geprüft, kritisch bewertet und überarbeitet.

\vspace{2em}
\noindent\textbf{Hinweis:} Alle generierten Vorschläge wurden eigenständig geprüft, fachlich bewertet und gegebenenfalls angepasst oder verworfen. Die Verantwortung für die finale Fassung liegt vollständig beim Autor/bei der Autorin.

\begin{landscape}
\begin{longtable}{|p{3cm}|p{3.5cm}|p{6cm}|p{5cm}|p{2.5cm}|}
\hline
\textbf{KI-Tool} & \textbf{Zweck} & \textbf{Kontext/Eingangstext} & \textbf{Generierte Ausgabe} & \textbf{Kapitel} \\
\hline
\endfirsthead

\multicolumn{5}{c}%
{{\tablename\ \thetable{} --- Fortsetzung}} \\
\hline
\textbf{KI-Tool} & \textbf{Zweck} & \textbf{Kontext/Eingangstext} & \textbf{Generierte Ausgabe} & \textbf{Kapitel} \\
\hline
\endhead

\hline
\endfoot

\hline
\endlastfoot

% Beispieleinträge (bitte anpassen oder löschen)
Anthropic Claude Opus 4.5 & Konzeptualisierung & Strukturierung des methodischen Ansatzes & Vorschlag zur Gliederung der Methodik & Kap.~3.1 \\
\hline

Anthropic Claude Opus 4.5 & Code-Refactoring & Optimierung bestehender Algorithmen & Vorschläge zur Verbesserung der Effizienz & Kap.~4.3 \\
\hline

Cohere Command A & RAG/Agentik & Entwurf eines Retrieval-Flows & Vorschlag für Tool-Aufrufe und Prompt-Struktur & Kap.~3.2 \\
\hline

DeepL Agent & Übersetzung & Deutsche Formulierung für Abstract & Professionelle englische Übersetzung & Abstract \\
\hline

GitHub Copilot & Code-Vervollständigung & Python-Funktion für Datenverarbeitung & Vorschläge für Funktionsimplementierung & Kap.~4.2 \\
\hline

Google Gemini 3 Pro & Literaturrecherche & Zusammenfassung aktueller Forschungsarbeiten & Überblick über Stand der Technik & Kap.~2.1 \\
\hline

Mistral Large 3 & Technische Zusammenfassung & Auswertung von API-/RFC-Dokumenten & Kompakte Zusammenfassung der Kernaussagen & Kap.~2.4 \\
\hline

OpenAI GPT-5.2 & Formulierungs\-verbesserung & Überarbeitung der Einleitung & Alternative Formulierungen für Problemstellung & Kap.~1.2, S.~5 \\
\hline

xAI Grok 4.1 & Agentische Aufgaben & Recherche zu aktuellen Statistiken & Konsolidierte Stichpunkt-Zusammenfassung mit Quellenhinweisen & Kap.~2.3 \\
\hline

% Weitere Einträge hier einfügen...
% Tool & Zweck & Kontext & Ausgabe & Kapitel \\
% \hline

\end{longtable}
\end{landscape}

\chapter{Zusätzliche Materialien}

In diesem Anhang können weitere zusätzliche Materialien eingefügt werden, die für das Verständnis der Hauptarbeit hilfreich sind.

\section{Weitere Tabellen und Daten}

Tabellarische Daten, die zu umfangreich für die Hauptkapitel sind.

\section{Quellcode und Implementierungsdetails}

Längere Quellcode-Listings oder detaillierte Implementierungen.

\section{Ergänzende Berechnungen}

Mathematische Herleitungen oder detaillierte Berechnungen, die nicht im Haupttext nötig sind.


%% Dokument ENDE %%%%%%%%%%%%%%%%%%%%%%%%%%%%%%%%%%%%%%%%%%%%%%%%%%%%%%%%%%
\end{document}
