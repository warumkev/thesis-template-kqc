% Placeholder macros/newcommands.tex
% Real macro definitions were not provided by the user.
% This file supplies safe, minimal no-op commands so the document can be built.

% Simple formatting helpers (no-ops or lightweight defaults)
\providecommand{\code}[1]{\texttt{#1}}
\providecommand{\TODO}[1]{\textbf{TODO: #1}}
\providecommand{\note}[1]{\textit{#1}}

% Ensure common metadata macros exist (if not defined elsewhere)
\providecommand{\theAuthor}{}
\providecommand{\theTitle}{}

% End of placeholder
%% ========================================================================
%% Benutzerdefinierte LaTeX-Befehle für Abschlussarbeiten
%% ========================================================================

%% Abbildungs- und Gleichungsverweise ====================================
\newcommand{\figureref}[1]{(Abbildung \ref{#1})}
\newcommand{\eqnref}[1]{(\ref{#1})}

%% Abbildungsüberschriften mit definierter Breite ========================
\newcommand{\wcaption}[2]{%
   \begin{minipage}{#1}%
   \caption{#2}%
   \end{minipage}%
}

%% Randnotizen mit optimierter Formatierung ===============================
\newcommand{\marginlabel}[1]{\marginnote{#1}}
\newcommand{\complex}{\mathbb{C}} % Complex
\newcommand{\real}{\mathbb{R}}    % Real
%\newcommand{\R}{\real}						% Real
%\newcommand{\N}{\mathbb{N}}
%\newcommand{\Z}{\mathbb{Z}}
\renewcommand{\L}{\mathcal{L}}
\newcommand{\N}{\mathcal{N}}
\newcommand{\R}{\mathcal{R}}
\newcommand{\D}{\mathcal{D}}
%
\newcommand{\Ham}{\mathcal{H}}    % Hamilton
\newcommand{\Prob}{\mathscr{P}}    % Hamilton
\newcommand{\unity}{\mathds{1}}   % Real
%

\newcommand\gammab{\gamma_\bot}
\newcommand\gammap{\gamma_\parallel}
\newcommand\gammai{\gamma_\text{int}}
\newcommand\gammae{\gamma_\text{ext}}

% -- New Operators --
\DeclareMathOperator{\rot}{rot}
\DeclareMathOperator{\grad}{grad}
%\DeclareMathOperator{\div}{div}
\renewcommand{\div}{\text{div}\,}
\DeclareMathOperator{\Tr}{Tr}
\DeclareMathOperator{\const}{const}
\DeclareMathOperator{\e}{e} 			% exponatial Function

% -- new symbols --
\newcommand{\laplace}{\Delta}
\newcommand{\dalembert}{\Box}

% -- new arrows --
\renewcommand{\leadsto}{\Longrightarrow}
\newcommand{\leftrightleadsto}{\Longleftrightarrow}


% -- Text subscripts--
\newcommand{\rel}{_\text{rel}}
%\newcommand{\st}{\text{st}}
%

% -- other --
\newcommand{\com}[2]{\underbrace{#1}_{\textrm{\scriptsize #2}}}
\newcommand{\with}[1]{\stackrel{\ref{#1}}{\Longrightarrow}}
%\newcommand{\unit}[1]{\,\textrm{#1}}

%\newcommand{\variance}[1]{\delta \mean{#1}^2}
\newcommand{\variance}[1]{(\Delta{#1})^2}
%\newcommand{\variance}[1]{\delta #1^2}

% -- Physics --------------------------------
\newcommand\op[1]{{\hat{\mathrm{#1}}}}  % Operator

\newcommand\expect[1]{\ensuremath{\left\langle{#1}\right\rangle}} %
%
\newcommand{\mean}[1]{\ensuremath{\overline{#1}}} % mean value
%
\newcommand{\state}[1]{\ensuremath{\ket{#1}}}
%
\newcommand\commutator[2]{\ensuremath{\mathinner{%
    \mathopen[\,#1,#2\,\mathclose]}}}
\newcommand{\Commutator}[2]{\ensuremath{\left[\,#1,#2\,\right]}}
\newcommand{\bigcommutator}[2]{\ensuremath{\bigl[\,#1,#2\,\bigr]}}
\newcommand{\Bigcommutator}[2]{\ensuremath{\Bigl[\,#1,#2\,\Bigr]}}
%
\newcommand\poisson[2]{\mathinner{%
    \mathopen\{#1,#2\mathclose\}}}
%

% -- Layout --------------------------------

\newcommand*{\dashfill}{\leavevmode\cleaders\hbox{-}\hfill\kern0pt}

\newcommand*{\midhrulefill}{
\leavevmode
\cleaders\hbox to 1ex{\raisebox{.5ex}{\rule{1ex}{.4pt}}}\hfill\kern0pt
}

%% Prozentangaben mit dünnem, geschütztem Abstand
%% Nutzung: \pct{20} ergibt "20 %" mit nicht trennbarem dünnem Leerzeichen
\newcommand{\pct}[1]{\SI{#1}{\percent}}

