% Placeholder macros/TableCommands.tex
% Minimal table-related helpers so the document compiles even without the real file.

\providecommand{\TableHeader}[1]{\textbf{#1}}
\newenvironment{SimpleTable}[1][]{% begin: do nothing special
}{% end: nothing
}

% no-op column type if referenced
\providecolumntype{P}[1]{>{\raggedright\arraybackslash}p{#1}}

% End of placeholder
%% ========================================================================
%% Befehle für professionelle Tabellenformatierung
%% ========================================================================
%% Basierend auf: The LaTeX Companion, tabsatz.ps und Best Practices

%% Tabellenfarben ===========================================================
\IfPackageLoaded{xcolor}{
   \colorlet{tableheadcolor}{gray!25}
   \colorlet{tablerowcolor}{gray!10}
}

%% Spaltendefinitionen für Tabellen ========================================
%% Neue Spaltentypen mit verbesserter Textformatierung

% Linksgebündige Spalte mit definierter Breite und Flattersatz
\newcommand{\PreserveBackslash}[1]{\let\temp=\\#1\let\\=\temp}
\newcolumntype{v}[1]{>{\PreserveBackslash\RaggedRight\hspace{0pt}}p{#1}}

% Mittelaligned Spalte mit Flattersatz
%% Zellformatierung in Tabellen ============================================
\newcommand{\removeindentation}{%
	\leftmargini=\labelsep
	\advance\leftmargini by \labelsep
}

\makeatletter
\newcommand\tableitemize{
	\@minipagetrue
	\removeindentation
}
\makeatother

%% Tabellenumgebung =========================================================
\newenvironment{Tabelle}[2][c]{%
  \tablestylecommon
  \begin{longtable}[#1]{#2}
  }
  {\end{longtable}%
  \tablerestoresettings
}

%% Formatierungsbefehle für Tabellen ========================================
\newcommand{\tablefontsize}{\footnotesize}
\newcommand{\tableheadfontsize}{\footnotesize}

% Standard-Tabellenformatierung
\newcommand\tablestylecommon{%
  \renewcommand{\arraystretch}{1.4}% Größere Zeilenabstände
  \normalfont\normalsize
  \sffamily\tablefontsize
  \centering
}

% Einstellungen zurücksetzen
\newcommand\tablerestoresettings{%
  \renewcommand{\arraystretch}{1}
  \normalsize\rmfamily
}

% Tabellenkopf-Formatierung
\newcommand\tablehead{%
  \tableheadfontsize\sffamily\bfseries
}

\newcommand\tableheadcolor{%
	\rowcolor{tableheadcolor}
}

\newcommand{\tableend}{\arrayrulecolor{black}\hline}
